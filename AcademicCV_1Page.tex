%%%%%%%%%%%%%%%%%%%%%%%%%%%%%%%%%%%%%%%%%%%%%%%%%%%%%%%%%%%%%%%%%%%%%%%%
%%%%%%%%%%%%%%%%%%%%%% Simple LaTeX CV Template %%%%%%%%%%%%%%%%%%%%%%%%
%%%%%%%%%%%%%%%%%%%%%%%%%%%%%%%%%%%%%%%%%%%%%%%%%%%%%%%%%%%%%%%%%%%%%%%%

%%%%%%%%%%%%%%%%%%%%%%%%%%%%%%%%%%%%%%%%%%%%%%%%%%%%%%%%%%%%%%%%%%%%%%%%
%% NOTE: If you find that it says                                     %%
%%                                                                    %%
%%                           1 of ??                                  %%
%%                                                                    %%
%% at the bottom of your first page, this means that the AUX file     %%
%% was not available when you ran LaTeX on this source. Simply RERUN  %%
%% LaTeX to get the ``??'' replaced with the number of the last page  %%
%% of the document. The AUX file will be generated on the first run   %%
%% of LaTeX and used on the second run to fill in all of the          %%
%% references.                                                        %%
%%%%%%%%%%%%%%%%%%%%%%%%%%%%%%%%%%%%%%%%%%%%%%%%%%%%%%%%%%%%%%%%%%%%%%%%

%%%%%%%%%%%%%%%%%%%%%%%%%%%% Document Setup %%%%%%%%%%%%%%%%%%%%%%%%%%%%

% Don't like 10pt? Try 11pt or 12pt
\documentclass[10pt]{article}
\RequirePackage[T1]{fontenc}
% LaTeX will typeset using Computer Modern Roman, which a lot of
% non-mathematicians and non-engineers won't like. Also, a few PDF
% viewers may not render CMR very well. Instead, Times New Roman can
% be used. That's what this package does.
\usepackage{times}

% The automated optical recognition software used to digitize resume
% information works best with fonts that do not have serifs. This
% command uses a sans serif font throughout. Uncomment both lines (or at
% least the second) to restore a Roman font (i.e., a font with serifs).
% (NOTE: This requires the times package above)
%\renewcommand{\familydefault}{\sfdefault}


% This is a helpful package that puts math inside length specifications
\usepackage{calc}

% This package helps LaTeX auto-hyphenate hyphenated words if you use
% special hyphens. For example, bio\-/mimicry will properly hyphenate
% ``mimicry'' if necessary.
\usepackage[shortcuts]{extdash}

% Layout: Puts the section titles on left side of page
\reversemarginpar

%
%         PAPER SIZE, PAGE NUMBER, AND DOCUMENT LAYOUT NOTES:
%
% The next \usepackage line changes the layout for CV style section
% headings as marginal notes. It also sets up the paper size as either
% letter or A4. By default, letter was used. If A4 paper is desired,
% comment out the letterpaper lines and uncomment the a4paper lines.
%
% As you can see, the margin widths and section title widths can be
% easily adjusted.
%
% ALSO: Notice that the includefoot option can be commented OUT in order
% to put the PAGE NUMBER *IN* the bottom margin. This will make the
% effective text area larger.
%
% IF YOU WISH TO REMOVE THE ``of LASTPAGE'' next to each page number,
% see the note about the +LP and -LP lines below. Comment out the +LP
% and uncomment the -LP.
%
% IF YOU WISH TO REMOVE PAGE NUMBERS, be sure that the includefoot line
% is uncommented and ALSO uncomment the \pagestyle{empty} a few lines
% below.
%

%% Use these lines for letter-sized paper
\usepackage[paper=a4paper,
            %includefoot, % Uncomment to put page number above margin
            marginparwidth=1.2in,     % Length of section titles
            marginparsep=.05in,       % Space between titles and text
            margin=1in,               % 1 inch margins
            includemp]{geometry}

%% Use these lines for A4-sized paper
%\usepackage[paper=a4paper,
%            %includefoot, % Uncomment to put page number above margin
%            marginparwidth=30.5mm,    % Length of section titles
%            marginparsep=1.5mm,       % Space between titles and text
%            margin=25mm,              % 25mm margins
%            includemp]{geometry}

%% More layout: Get rid of indenting throughout entire document
\setlength{\parindent}{0in}

% Provides special list environments and macros to create new ones
\usepackage[shortlabels]{enumitem}

% Simpler bibsections for CV sections
% (thanks to natbib for inspiration)
%
% * For lists of references with hanging indents and no numbers:
%
%   \begin{bibsection}
%       \item ...
%   \end{bibsection}
%
% * For numbered lists of references (with hanging indents):
%
%   \begin{bibenum}
%       \item ...
%   \end{bibenum}
%
%   Note that bibenum numbers continuously throughout. To reset the
%   counter, use
%
%   \restartlist{bibenum}
%
%   at the place where you want the numbering to reset.

\makeatletter
\newlength{\bibhang}
\setlength{\bibhang}{1em}
\newlength{\bibsep}
 {\@listi \global\bibsep\itemsep \global\advance\bibsep by\parsep}
\newlist{bibsection}{itemize}{3}
\setlist[bibsection]{label=,leftmargin=\bibhang,%
        itemindent=-\bibhang,
        itemsep=\bibsep,parsep=\z@,partopsep=0pt,
        topsep=0pt}
\newlist{bibenum}{enumerate}{3}
\setlist[bibenum]{label=[\arabic*],resume,leftmargin={\bibhang+\widthof{[999]}},%
        itemindent=-\bibhang,
        itemsep=\bibsep,parsep=\z@,partopsep=0pt,
        topsep=0pt}
\let\oldendbibenum\endbibenum
\def\endbibenum{\oldendbibenum\vspace{-.6\baselineskip}}
\let\oldendbibsection\endbibsection
\def\endbibsection{\oldendbibsection\vspace{-.6\baselineskip}}
\makeatother

%% Reference the last page in the page number
%
% NOTE: comment the +LP line and uncomment the -LP line to have page
%       numbers without the ``of ##'' last page reference)
%
% NOTE: uncomment the \pagestyle{empty} line to get rid of all page
%       numbers (make sure includefoot is commented out above)
%
\usepackage{fancyhdr,lastpage}
\pagestyle{fancy}
%\pagestyle{empty}      % Uncomment this to get rid of page numbers
\fancyhf{}\renewcommand{\headrulewidth}{0pt}
\fancyfootoffset{\marginparsep+\marginparwidth}
\newlength{\footpageshift}
\setlength{\footpageshift}
          {0.5\textwidth+0.5\marginparsep+0.5\marginparwidth-2in}
%\lfoot{\hspace{\footpageshift}%
%       \parbox{4in}{\, \hfill %
%                    \arabic{page} of \protect\pageref*{LastPage} % +LP
%%                    \arabic{page}                               % -LP
%                    \hfill \,}}

% Finally, give us PDF bookmarks
\usepackage{color,hyperref}
\definecolor{darkblue}{rgb}{0.0,0.0,0.3}
\hypersetup{colorlinks,breaklinks,
            linkcolor=darkblue,urlcolor=darkblue,
            anchorcolor=darkblue,citecolor=darkblue}
\newcommand{\MYhref}[3][blue]{\href{#2}{\color{#1}{#3}}}%
%%%%%%%%%%%%%%%%%%%%%%%% End Document Setup %%%%%%%%%%%%%%%%%%%%%%%%%%%%


%%%%%%%%%%%%%%%%%%%%%%%%%%% Helper Commands %%%%%%%%%%%%%%%%%%%%%%%%%%%%

%%% HEADING AT TOP OF CURRICULUM VITAE

% The title (name) with a horizontal rule under it
% (optional argument typesets an object right-justified across from name
%  as well)
%
% Usage: \makeheading{name}
%        OR
%        \makeheading[right_object]{name}
%
% Place at top of document. It should be the first thing.
% If ``right_object'' is provided in the square-braced optional
% argument, it will be right justified on the same line as ``name'' at
% the top of the CV. For example:
%
%       \makeheading[\emph{Curriculum vitae}]{Your Name}
%
% will put an emphasized ``Curriculum vitae'' at the top of the document
% as a title. Likewise, a picture could be included:
%
%   \makeheading[{\includegraphics[height=1.5in]{my_picture}}]{Your Name}
%
% the picture will be flush right across from the name. For this example
% to work, make sure the extra set of curly braces is included. Also
% makes ure that \usepackage{graphicx} is somewhere in the preamble.
\newcommand{\makeheading}[2][]%
        {\hspace*{-\marginparsep minus \marginparwidth}%
         \begin{minipage}[t]{\textwidth+\marginparwidth+\marginparsep}%
             {\large \bfseries #2 \hfill #1}\\[-0.15\baselineskip]%
                 \rule{\columnwidth}{1pt}%
         \end{minipage}}

%%% SECTION HEADINGS

% The section headings. Flush left in small caps down pseudo-margin.
%
% Usage: \section{section name}
\renewcommand{\section}[1]{\pagebreak[3]%
    \vspace{1.3\baselineskip}%
    \phantomsection\addcontentsline{toc}{section}{#1}%
    \noindent\llap{\scshape\smash{\parbox[t]{\marginparwidth}{\hyphenpenalty=10000\raggedright #1}}}%
    \vspace{-\baselineskip}\par}

%%% LISTS

% This macro alters a list by removing some of the space that follows the list
% (is used by lists below)
\newcommand*\fixendlist[1]{%
    \expandafter\let\csname preFixEndListend#1\expandafter\endcsname\csname end#1\endcsname
    \expandafter\def\csname end#1\endcsname{\csname preFixEndListend#1\endcsname\vspace{-0.6\baselineskip}}}

% These macros help ensure that items in outer-type lists do not get
% separated from the next line by a page break
% (they are used by lists below)
\let\originalItem\item
\newcommand*\fixouterlist[1]{%
    \expandafter\let\csname preFixOuterList#1\expandafter\endcsname\csname #1\endcsname
    \expandafter\def\csname #1\endcsname{\let\oldItem\item\def\item{\pagebreak[2]\oldItem}\csname preFixOuterList#1\endcsname}
    \expandafter\let\csname preFixOuterListend#1\expandafter\endcsname\csname end#1\endcsname
    \expandafter\def\csname end#1\endcsname{\let\item\oldItem\csname preFixOuterListend#1\endcsname}}
\newcommand*\fixinnerlist[1]{%
    \expandafter\let\csname preFixInnerList#1\expandafter\endcsname\csname #1\endcsname
    \expandafter\def\csname #1\endcsname{\let\oldItem\item\let\item\originalItem\csname preFixInnerList#1\endcsname}
    \expandafter\let\csname preFixInnerListend#1\expandafter\endcsname\csname end#1\endcsname
    \expandafter\def\csname end#1\endcsname{\csname preFixInnerListend#1\endcsname\let\item\oldItem}}

% An itemize-style list with lots of space between items
%
% Usage:
%   \begin{outerlist}
%       \item ...    % (or \item[] for no bullet)
%   \end{outerlist}
\newlist{outerlist}{itemize}{3}
    \setlist[outerlist]{label=\enskip\textbullet,leftmargin=*}
    \fixendlist{outerlist}
    \fixouterlist{outerlist}

% An environment IDENTICAL to outerlist that has better pre-list spacing
% when used as the first thing in a \section
%
% Usage:
%   \begin{lonelist}
%       \item ...    % (or \item[] for no bullet)
%   \end{lonelist}
\newlist{lonelist}{itemize}{3}
    \setlist[lonelist]{label=\enskip\textbullet,leftmargin=*,partopsep=0pt,topsep=0pt}
    \fixendlist{lonelist}
    \fixouterlist{lonelist}

% An itemize-style list with little space between items
%
% Usage:
%   \begin{innerlist}
%       \item ...    % (or \item[] for no bullet)
%   \end{innerlist}
\newlist{innerlist}{itemize}{3}
    \setlist[innerlist]{label=\enskip\textbullet,leftmargin=*,parsep=0pt,itemsep=0pt,topsep=0pt,partopsep=0pt}
    \fixinnerlist{innerlist}

% An environment IDENTICAL to innerlist that has better pre-list spacing
% when used as the first thing in a \section
%
% Usage:
%   \begin{loneinnerlist}
%       \item ...    % (or \item[] for no bullet)
%   \end{loneinnerlist}
\newlist{loneinnerlist}{itemize}{3}
    \setlist[loneinnerlist]{label=\enskip\textbullet,leftmargin=*,parsep=0pt,itemsep=0pt,topsep=0pt,partopsep=0pt}
    \fixendlist{loneinnerlist}
    \fixinnerlist{loneinnerlist}

%%% EXTRA SPACE

% To add some paragraph space between lines.
% This also tells LaTeX to preferably break a page on one of these gaps
% if there is a needed pagebreak nearby.
\newcommand{\blankline}{\quad\pagebreak[3]}
\newcommand{\halfblankline}{\quad\vspace{-0.5\baselineskip}\pagebreak[3]}

%%% FORMATTING MACROS

% Provides a linked \doi{#1} that links doi:#1 to http://dx.doi.org/#1
\usepackage{doi}
% To change the text before the DOI, adjust this command
%\renewcommand\doitext{doi:}

% Provides a linked \url{#1} that doesn't require escape characters
\usepackage{url}

% You can adjust the style \url{} uses here:
% (options are: same, rm, sf, tt; defaults to tt)
\urlstyle{same}

% For \email{ADDRESS}, links ADDRESS to the url mailto:ADDRESS
% (uncomment to typeset the e\-/mail address in typewriter font;
%  otherwise, will be typeset in the \urlstyle above)
%\DeclareUrlCommand\emaillink{\urlstyle{tt}}
\providecommand*\emaillink[1]{\nolinkurl{#1}}
\providecommand*\email[1]{\href{mailto:#1}{\emaillink{#1}}}

\providecommand\BibTeX{{B\kern-.05em{\sc i\kern-.025em b}\kern-.08em \TeX}}
\providecommand\Matlab{\textsc{Matlab}}

% Custom hyphenation rules for words that LaTeX has trouble with
\hyphenation{bio-mim-ic-ry bio-in-spi-ra-tion re-us-a-ble pro-vid-er Media-Wiki}

%%%%%%%%%%%%%%%%%%%%%%%% End Helper Commands %%%%%%%%%%%%%%%%%%%%%%%%%%%

%%%%%%%%%%%%%%%%%%%%%%%%% Begin CV Document %%%%%%%%%%%%%%%%%%%%%%%%%%%%

\begin{document}
\makeheading{Paul James Wright}

\section{Contact Information}

% NOTE: Mind where the & separators and \\ breaks are in the following
%       table. Table is one row made up of three parboxes. The left
%       parbox has address info, the middle parbox has a vertical bar,
%       and the right parbox has phone and electronic contact
%       information.
%
% MACROS: \rcollength is the width of the right column of the table
%             (adjust it to your liking; default is 1.85in).
%         \spacewidth is width of area between left and right boxes.
%
%\newlength{\rcollength}\setlength{\rcollength}{0in}%
\newlength{\spacewidth}\setlength{\spacewidth}{20pt}
%
%\begin{tabular}[t]{@{}p{\textwidth-\rcollength-\spacewidth}@{}p{\spacewidth}@{}p{\rcollength}}%

{Rm~614, Kelvin Building \hfill Work: +44 (0)14133 08855 \\
University of Glasgow \hfill{Web: \href{http://www.pauljwright.co.uk}{www.pauljwright.co.uk}} \\ 
Glasgow, G12 8QQ \hfill{Email: paul.wright@glasgow.ac.uk} \\ 
United Kingdom \hfill{Publication List: \href{https://ui.adsabs.harvard.edu/#search/q=orcid\%3A0000-0001-9021-611X&sort=date+desc}{SAO/NASA ADS}}}

%\hfill{GitHub: \href{http://www.github.com/pauljwright}{www.github.com/pauljwright}} }


%\end{tabular}

%%
%% In modern CV's, it seems like ``Objective'' is frowned upon. Instead,
%% incorporate it into a well-constructed cover letter. The ``More
%% information'' can go at the end of the CV, but it should not distract
%% from the section giving references available to contact.
%%
%
% \section{Objective}
%
% Placement in an academic position (i.e., faculty, postdoctoral, or
% research scientist) that allows for advanced research in distributed
% complex adaptive systems (i.e., modeling, analysis, design, and
% verification) with a particular focus on the control of engineered
% agents (e.g., for communications, control, software, electronics, and
% sustainability) and the analysis of biological phenomena (e.g.,
% self-organization, ecological rationality)
% \begin{innerlist}
% \item More information and auxiliary documents can be found at\\\url{http://www.tedpavlic.com/facjobsearch/}
% \end{innerlist}

\section{Research Interests}

My interests range from stellar to solar physics; my main interests lie in the heating of the solar atmosphere, including active regions and loops. Currently I am gaining expertise in analysis of data from \textit{SDO}/AIA, \textit{Hinode}/XRT, \textit{Hinode}/EIS and \textit{NuSTAR}. 

\section{Current Academic Appointment}

\textbf{Post-Graduate Research Assistant},
            \href{http://www.gla.ac.uk/}{University of Glasgow}
            \hfill {2014 -- present}
\begin{innerlist}

    \item[] \href{http://astro.gla.ac.uk/}{SUPA School of Physics and Astronomy}
    \begin{innerlist}
    	\item[] Project: \emph{The Energetics of Small Flares and Brightenings}
    \end{innerlist}

\end{innerlist}
 
\section{Previous Academic Appointments}
 
\textbf{Visiting Researcher},
	\href{}{NASA Goddard Space Flight Center (GSFC)} \hfill {2016}
\begin{innerlist}

    \item[] \href{https://science.gsfc.nasa.gov/heliophysics/}{Heliophysics Science Division}
    \begin{innerlist}
    \item[] Collaborators: {\it Nicholeen Viall, Jack Ireland}
    \end{innerlist}
    
\end{innerlist}

\halfblankline

 \textbf{Research Scholar},
	\href{}{Harvard-Smithsonian Center for Astrophysics (CfA)} \hfill {2013 -- 2014}
\begin{innerlist}

    \item[] \href{https://www.cfa.harvard.edu/research/hea/sun}{Solar and Stellar X-Ray Group}
    \begin{innerlist}
    	\item[] Designed and implemented a sophisticated stellar flare detection routine for long-cadence (30 mins) {\it Kepler} data. 
    	\item[] Collaborators: {\it Steven Saar, S{\o}ren Meibom, Jeremy Drake, Vinay Kashyap}
    \end{innerlist}

\end{innerlist}

%%%% EDUCATION %%%%
\section{Education}

\href{http://www.gla.ac.uk/}{\textbf{University of Glasgow}},
Glasgow, UK \hfill {2014 -- present}
\begin{innerlist}
\item[] Ph.D.
        \href{}
             {Solar Physics}
        \begin{innerlist}
        \item[] Thesis Topic: \emph{The Energetics of Small Flares and Brightenings}
        \item[] Advisers:
              \href{}
                   {Dr Iain G. Hannah, Dr Alexander MacKinnon}
        \end{innerlist}

\end{innerlist}

\halfblankline

\href{http://www.soton.ac.uk/}{\textbf{University of Southampton}},
Southampton, UK \hfill {2010 -- 2014}
\begin{innerlist}

\item[] MPhys
        \href{http://www.phys.soton.ac.uk/programmes/f3fm-mphys-astronomy-year-abroad-4-yrs/}
             {Astrophysics with a year abroad}
        \begin{innerlist}
        \item[] First class honours (1:1)
         \item[] Adviser:
              \href{}
                   {Professor Malcolm Coe}
        \end{innerlist}

\end{innerlist}

\halfblankline

\href{https://www.cfa.harvard.edu/}{\textbf{Harvard-Smithsonian CfA/Harvard University}},
Cambridge, MA \hfill {2013 -- 2014}
\begin{innerlist}

\item[] MPhys
        \href{http://www.phys.soton.ac.uk/programmes/f3fm-mphys-astronomy-year-abroad-4-yrs/}
             {Astrophysics with a year abroad}
        \begin{innerlist}
        \item[] Thesis Topic: \emph{Superflare Rates of Solar-Like Stars}
        \item[] Advisers:
              \href{}
                   {Dr Steven H. Saar, Dr Jeremy J. Drake}
        \end{innerlist}

\end{innerlist}

\section{Refereed Journal Publications}

\begin{bibenum}
	\item{} Wang, J., Sim{\~o}es, P.~J.~A., Jeffrey, N.~L.~S. \textit{et al} 2017. {\it Observations of Reconnection Flows in a Flare on The Solar Disk} 2017, in revision %\MYhref[magenta]{...}{ApJL}, \MYhref[blue]{...}{xxx, xxx}

	\item{} Wright, P. J., Hannah, I.~G., Grefenstette, B. W., \textit{et al} 2017. {\it Microflare Heating of a Solar Active Region Observed with {\it NuSTAR}, {\it Hinode}/XRT, and {\it SDO}/AIA}, \MYhref[magenta]{https://doi.org/10.3847/1538-4357/aa7a59}{ApJ}, \href{http://adsabs.harvard.edu/abs/2017ApJ...844..132W}{844, 132}

	\item{} Kuhar, M., Krucker, S., Hannah, I. G., \textit{et al} 2017. {\it Evidence of Significant Energy Input in the Late Phase of a Solar Flare from NuSTAR X-ray Observations}, \MYhref[magenta]{http://doi:10.3847/1538-4357/835/1/6}{ApJ}, \href{http://adsabs.harvard.edu/abs/2017ApJ...835....6K}{835, 6}

\end{bibenum}

\section{Papers in Prep.}

\begin{bibenum}
	\item{} Wright, P. J., Saar, S.~H., Meibom, S., \textit{et al.} {\bf [title redacted]} 2017
\end{bibenum}

\section{Select Awards}

\href{https://spd.aas.org/prizes/studentship}{Solar Physics Division Meeting (SPD) Studentship Award} \hfill 2017

{\it Cumulative total of Awards and Grants:} \textsterling7000

%\section{Outreach}
%
%\begin{innerlist}
%\item[] {\bf \href{http://www.glasgowsciencecentre.org/}{Glasgow Science Centre}}, Demonstrator \hfill 2016
%\item[] {\bf \href{https://www.britishscienceweek.org/}{British Science Week}}, Demonstrator \hfill 2016
%\item[] {\bf \href{http://www.gla.ac.uk/explore/scienceconnects/news/headline_451512_en.html}{Institute of Physics: {\it Women and Girls in Science}}}, Demonstrator \hfill 2016
%\item[] {\bf \href{http://www.stv.tv/}{Scottish Television (STV)}}, Guest Presenter \hfill 2015
%\item[] {\bf \href{http://www.worldwidetelescope.org/}{World Wide Telescope}}, Ambassador \hfill 2013 -- 2014
%\item[] {\bf \href{http://www.bbc.co.uk/programmes/b019h4g8}{BBC Stargazing Live}}, Demonstrator \hfill 2013
%\item[] {\bf \href{http://www.southampton.ac.uk/~sj2y10/}{So'ton Astrodome}}, Demonstrator \hfill 2012
%\item[] {\bf \href{http://www.bbc.co.uk/programmes/b00lwxj1}{BBC Bang Goes The Theory Roadshow}}, Demonstrator \hfill 2012
%\end{innerlist}


\section{Select Memberships}
     \textbf{International Space Science Institute (ISSI)}, Young Scientist Member \hfill {2015 -- present}
    \begin{innerlist}
     \item[] Member of Paola Testa's ISSI Team: \href{http://helio.cfa.harvard.edu/ISSI/welcome_pt.html}{\textit{New Diagnostics of Particle Acceleration in Solar Coronal Nanoflares from Chromospheric Observations and Modeling}}
    \end{innerlist}

\end{document}

%%%%%%%%%%%%%%%%%%%%%%%%%% End CV Document %%%%%%%%%%%%%%%%%%%%%%%%%%%%%

%----------------------------------------------------------------------%
% The following is copyright and licensing information for
% redistribution of this LaTeX source code; it also includes a liability
% statement. If this source code is not being redistributed to others,
% it may be omitted. It has no effect on the function of the above code.
%----------------------------------------------------------------------%
% Copyright (c) 2007, 2008, 2009, 2010, 2011 by Theodore P. Pavlic
%
% Unless otherwise expressly stated, this work is licensed under the
% Creative Commons Attribution-Noncommercial 3.0 United States License. To
% view a copy of this license, visit
% http://creativecommons.org/licenses/by-nc/3.0/us/ or send a letter to
% Creative Commons, 171 Second Street, Suite 300, San Francisco,
% California, 94105, USA.
%
% THE SOFTWARE IS PROVIDED "AS IS", WITHOUT WARRANTY OF ANY KIND, EXPRESS
% OR IMPLIED, INCLUDING BUT NOT LIMITED TO THE WARRANTIES OF
% MERCHANTABILITY, FITNESS FOR A PARTICULAR PURPOSE AND NONINFRINGEMENT.
% IN NO EVENT SHALL THE AUTHORS OR COPYRIGHT HOLDERS BE LIABLE FOR ANY
% CLAIM, DAMAGES OR OTHER LIABILITY, WHETHER IN AN ACTION OF CONTRACT,
% TORT OR OTHERWISE, ARISING FROM, OUT OF OR IN CONNECTION WITH THE
% SOFTWARE OR THE USE OR OTHER DEALINGS IN THE SOFTWARE.
%----------------------------------------------------------------------%