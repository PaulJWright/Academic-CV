%%%%%%%%%%%%%%%%%%%%%%%%%%%%%%%%%%%%%%%%%%%%%%%%%%%%%%%%%%%%%%%%%%%%%%%%
%%%%%%%%%%%%%%%%%%%%%% Simple LaTeX CV Template %%%%%%%%%%%%%%%%%%%%%%%%
%%%%%%%%%%%%%%%%%%%%%%%%%%%%%%%%%%%%%%%%%%%%%%%%%%%%%%%%%%%%%%%%%%%%%%%%

%%%%%%%%%%%%%%%%%%%%%%%%%%%%%%%%%%%%%%%%%%%%%%%%%%%%%%%%%%%%%%%%%%%%%%%%
%% NOTE: If you find that it says                                     %%
%%                                                                    %%
%%                           1 of ??                                  %%
%%                                                                    %%
%% at the bottom of your first page, this means that the AUX file     %%
%% was not available when you ran LaTeX on this source. Simply RERUN  %%
%% LaTeX to get the ``??'' replaced with the number of the last page  %%
%% of the document. The AUX file will be generated on the first run   %%
%% of LaTeX and used on the second run to fill in all of the          %%
%% references.                                                        %%
%%%%%%%%%%%%%%%%%%%%%%%%%%%%%%%%%%%%%%%%%%%%%%%%%%%%%%%%%%%%%%%%%%%%%%%%

%%%%%%%%%%%%%%%%%%%%%%%%%%%% Document Setup %%%%%%%%%%%%%%%%%%%%%%%%%%%%

% Don't like 10pt? Try 11pt or 12pt
\documentclass[11pt]{article}
\RequirePackage[T1]{fontenc}

% LaTeX will typeset using Computer Modern Roman, which a lot of
% non-mathematicians and non-engineers won't like. Also, a few PDF
% viewers may not render CMR very well. Instead, Times New Roman can
% be used. That's what this package does.
\usepackage{times}

% The automated optical recognition software used to digitize resume
% information works best with fonts that do not have serifs. This
% command uses a sans serif font throughout. Uncomment both lines (or at
% least the second) to restore a Roman font (i.e., a font with serifs).
% (NOTE: This requires the times package above)
%\renewcommand{\familydefault}{\sfdefault}

% This is a helpful package that puts math inside length specifications
\usepackage{calc}

% This package helps LaTeX auto-hyphenate hyphenated words if you use
% special hyphens. For example, bio\-/mimicry will properly hyphenate
% ``mimicry'' if necessary.
\usepackage[shortcuts]{extdash}

% Layout: Puts the section titles on left side of page
\reversemarginpar

%
%         PAPER SIZE, PAGE NUMBER, AND DOCUMENT LAYOUT NOTES:
%
% The next \usepackage line changes the layout for CV style section
% headings as marginal notes. It also sets up the paper size as either
% letter or A4. By default, letter was used. If A4 paper is desired,
% comment out the letterpaper lines and uncomment the a4paper lines.
%
% As you can see, the margin widths and section title widths can be
% easily adjusted.
%
% ALSO: Notice that the includefoot option can be commented OUT in order
% to put the PAGE NUMBER *IN* the bottom margin. This will make the
% effective text area larger.
%
% IF YOU WISH TO REMOVE THE ``of LASTPAGE'' next to each page number,
% see the note about the +LP and -LP lines below. Comment out the +LP
% and uncomment the -LP.
%
% IF YOU WISH TO REMOVE PAGE NUMBERS, be sure that the includefoot line
% is uncommented and ALSO uncomment the \pagestyle{empty} a few lines
% below.
%

%% Use these lines for letter-sized paper
\usepackage[paper=a4paper,
            %includefoot, % Uncomment to put page number above margin
            marginparwidth=1.2in,     % Length of section titles
            marginparsep=.05in,       % Space between titles and text
            margin=0.6in,               % 1 inch margins
            includemp]{geometry}

%% Use these lines for A4-sized paper
%\usepackage[paper=a4paper,
%            %includefoot, % Uncomment to put page number above margin
%            marginparwidth=30.5mm,    % Length of section titlesa
%            marginparsep=1.5mm,       % Space between titles and text
%            margin=25mm,              % 25mm margins
%            includemp]{geometry}

%% More layout: Get rid of indenting throughout entire document
\setlength{\parindent}{0in}

% Provides special list environments and macros to create new ones
\usepackage[shortlabels]{enumitem}

% Simpler bibsections for CV sections
% (thanks to natbib for inspiration)
%
% * For lists of references with hanging indents and no numbers:
%
%   \begin{bibsection}
%       \item ...
%   \end{bibsection}
%
% * For numbered lists of references (with hanging indents):
%
%   \begin{bibenum}
%       \item ...
%   \end{bibenum}
%
%   Note that bibenum numbers continuously throughout. To reset the
%   counter, use
%
%   \restartlist{bibenum}
%
%   at the place where you want the numbering to reset.

\makeatletter
\newlength{\bibhang}
\setlength{\bibhang}{1em}
\newlength{\bibsep}
 {\@listi \global\bibsep\itemsep \global\advance\bibsep by\parsep}
\newlist{bibsection}{itemize}{3}
\setlist[bibsection]{label=,leftmargin=\bibhang,%
        itemindent=-\bibhang,
        itemsep=\bibsep,parsep=\z@,partopsep=0pt,
        topsep=0pt}
\newlist{bibenum}{enumerate}{3}
\setlist[bibenum]{label=[\arabic*],resume,leftmargin={\bibhang+\widthof{[999]}},%
        itemindent=-\bibhang,
        itemsep=\bibsep,parsep=\z@,partopsep=0pt,
        topsep=0pt}
\let\oldendbibenum\endbibenum
\def\endbibenum{\oldendbibenum\vspace{-.6\baselineskip}}
\let\oldendbibsection\endbibsection
\def\endbibsection{\oldendbibsection\vspace{-.6\baselineskip}}
\makeatother

%% Reference the last page in the page number
%
% NOTE: comment the +LP line and uncomment the -LP line to have page
%       numbers without the ``of ##'' last page reference)
%
% NOTE: uncomment the \pagestyle{empty} line to get rid of all page
%       numbers (make sure includefoot is commented out above)
%
\usepackage{fancyhdr,lastpage}
\pagestyle{fancy}
%\pagestyle{empty}      % Uncomment this to get rid of page numbers
\fancyhf{}\renewcommand{\headrulewidth}{0pt}
\fancyfootoffset{\marginparsep+\marginparwidth}
\newlength{\footpageshift}
\setlength{\footpageshift}
          {0.5\textwidth+0.5\marginparsep+0.5\marginparwidth-2in}
\lfoot{\hspace{\footpageshift}%
       \parbox{4in}{\, \hfill %
                    \arabic{page} of \protect\pageref*{LastPage} % +LP
%                    \arabic{page}                               % -LP
                    \hfill \,}}

% Finally, give us PDF bookmarks
\usepackage{color,hyperref}
\definecolor{darkblue}{rgb}{0.0,0.0,0.3}
\hypersetup{colorlinks,breaklinks,
            linkcolor=darkblue,urlcolor=darkblue,
            anchorcolor=darkblue,citecolor=darkblue}
\newcommand{\MYhref}[3][blue]{\href{#2}{\color{#1}{#3}}}%
%%%%%%%%%%%%%%%%%%%%%%%% End Document Setup %%%%%%%%%%%%%%%%%%%%%%%%%%%%


%%%%%%%%%%%%%%%%%%%%%%%%%%% Helper Commands %%%%%%%%%%%%%%%%%%%%%%%%%%%%

%%% HEADING AT TOP OF CURRICULUM VITAE

% The title (name) with a horizontal rule under it
% (optional argument typesets an object right-justified across from name
%  as well)
%
% Usage: \makeheading{name}
%        OR
%        \makeheading[right_object]{name}
%
% Place at top of document. It should be the first thing.
% If ``right_object'' is provided in the square-braced optional
% argument, it will be right justified on the same line as ``name'' at
% the top of the CV. For example:
%
%       \makeheading[\emph{Curriculum vitae}]{Your Name}
%
% will put an emphasized ``Curriculum vitae'' at the top of the document
% as a title. Likewise, a picture could be included:
%
%   \makeheading[{\includegraphics[height=1.5in]{my_picture}}]{Your Name}
%
% the picture will be flush right across from the name. For this example
% to work, make sure the extra set of curly braces is included. Also
% makes ure that \usepackage{graphicx} is somewhere in the preamble.
\newcommand{\makeheading}[2][]%
        {\hspace*{-\marginparsep minus \marginparwidth}%
         \begin{minipage}[t]{\textwidth+\marginparwidth+\marginparsep}%
             {\large \bfseries #2 \hfill #1}\\[-0.15\baselineskip]%
                 \rule{\columnwidth}{1pt}%
         \end{minipage}}

%%% SECTION HEADINGS

% The section headings. Flush left in small caps down pseudo-margin.
%
% Usage: \section{section name}
\renewcommand{\section}[1]{\pagebreak[3]%
    \vspace{1.3\baselineskip}%
    \phantomsection\addcontentsline{toc}{section}{#1}%
    \noindent\llap{\scshape\smash{\parbox[t]{\marginparwidth}{\hyphenpenalty=10000\raggedright #1}}}%
    \vspace{-\baselineskip}\par}

%%% LISTS

% This macro alters a list by removing some of the space that follows the list
% (is used by lists below)
\newcommand*\fixendlist[1]{%
    \expandafter\let\csname preFixEndListend#1\expandafter\endcsname\csname end#1\endcsname
    \expandafter\def\csname end#1\endcsname{\csname preFixEndListend#1\endcsname\vspace{-0.6\baselineskip}}}

% These macros help ensure that items in outer-type lists do not get
% separated from the next line by a page break
% (they are used by lists below)
\let\originalItem\item
\newcommand*\fixouterlist[1]{%
    \expandafter\let\csname preFixOuterList#1\expandafter\endcsname\csname #1\endcsname
    \expandafter\def\csname #1\endcsname{\let\oldItem\item\def\item{\pagebreak[2]\oldItem}\csname preFixOuterList#1\endcsname}
    \expandafter\let\csname preFixOuterListend#1\expandafter\endcsname\csname end#1\endcsname
    \expandafter\def\csname end#1\endcsname{\let\item\oldItem\csname preFixOuterListend#1\endcsname}}
\newcommand*\fixinnerlist[1]{%
    \expandafter\let\csname preFixInnerList#1\expandafter\endcsname\csname #1\endcsname
    \expandafter\def\csname #1\endcsname{\let\oldItem\item\let\item\originalItem\csname preFixInnerList#1\endcsname}
    \expandafter\let\csname preFixInnerListend#1\expandafter\endcsname\csname end#1\endcsname
    \expandafter\def\csname end#1\endcsname{\csname preFixInnerListend#1\endcsname\let\item\oldItem}}

% An itemize-style list with lots of space between items
%
% Usage:
%   \begin{outerlist}
%       \item ...    % (or \item[] for no bullet)
%   \end{outerlist}
\newlist{outerlist}{itemize}{3}
    \setlist[outerlist]{label=\enskip\textbullet,leftmargin=*}
    \fixendlist{outerlist}
    \fixouterlist{outerlist}

% An environment IDENTICAL to outerlist that has better pre-list spacing
% when used as the first thing in a \section
%
% Usage:
%   \begin{lonelist}
%       \item ...    % (or \item[] for no bullet)
%   \end{lonelist}
\newlist{lonelist}{itemize}{3}
    \setlist[lonelist]{label=\enskip\textbullet,leftmargin=*,partopsep=0pt,topsep=0pt}
    \fixendlist{lonelist}
    \fixouterlist{lonelist}

% An itemize-style list with little space between items
%
% Usage:
%   \begin{innerlist}
%       \item ...    % (or \item[] for no bullet)
%   \end{innerlist}
\newlist{innerlist}{itemize}{3}
    \setlist[innerlist]{label=\enskip\textbullet,leftmargin=*,parsep=0pt,itemsep=0pt,topsep=0pt,partopsep=0pt}
    \fixinnerlist{innerlist}

% An environment IDENTICAL to innerlist that has better pre-list spacing
% when used as the first thing in a \section
%
% Usage:
%   \begin{loneinnerlist}
%       \item ...    % (or \item[] for no bullet)
%   \end{loneinnerlist}
\newlist{loneinnerlist}{itemize}{3}
    \setlist[loneinnerlist]{label=\enskip\textbullet,leftmargin=*,parsep=0pt,itemsep=0pt,topsep=0pt,partopsep=0pt}
    \fixendlist{loneinnerlist}
    \fixinnerlist{loneinnerlist}

%%% EXTRA SPACE

% To add some paragraph space between lines.
% This also tells LaTeX to preferably break a page on one of these gaps
% if there is a needed pagebreak nearby.
\newcommand{\blankline}{\quad\pagebreak[3]}
\newcommand{%\halfblankline}{\quad\vspace{-0.5\baselineskip}\pagebreak[3]}

%%% FORMATTING MACROS

% Provides a linked \doi{#1} that links doi:#1 to http://dx.doi.org/#1
\usepackage{doi}
% To change the text before the DOI, adjust this command
%\renewcommand\doitext{doi:}

% Provides a linked \url{#1} that doesn't require escape characters
\usepackage{url}

% You can adjust the style \url{} uses here:
% (options are: same, rm, sf, tt; defaults to tt)
\urlstyle{same}

% For \email{ADDRESS}, links ADDRESS to the url mailto:ADDRESS
% (uncomment to typeset the e\-/mail address in typewriter font;
%  otherwise, will be typeset in the \urlstyle above)
%\DeclareUrlCommand\emaillink{\urlstyle{tt}}
\providecommand*\emaillink[1]{\nolinkurl{#1}}
\providecommand*\email[1]{\href{mailto:#1}{\emaillink{#1}}}

\providecommand\BibTeX{{B\kern-.05em{\sc i\kern-.025em b}\kern-.08em \TeX}}
\providecommand\Matlab{\textsc{Matlab}}

% Custom hyphenation rules for words that LaTeX has trouble with
\hyphenation{bio-mim-ic-ry bio-in-spi-ra-tion re-us-a-ble pro-vid-er Media-Wiki}

%%%%%%%%%%%%%%%%%%%%%%%% End Helper Commands %%%%%%%%%%%%%%%%%%%%%%%%%%%

%%%%%%%%%%%%%%%%%%%%%%%%% Begin CV Document %%%%%%%%%%%%%%%%%%%%%%%%%%%%

\begin{document}
\makeheading{Paul James Wright}

\section{Contact Information}

% NOTE: Mind where the & separators and \\ breaks are in the following
%       table. Table is one row made up of three parboxes. The left
%       parbox has address info, the middle parbox has a vertical bar,
%       and the right parbox has phone and electronic contact
%       information.
%
% MACROS: \rcollength is the width of the right column of the table
%             (adjust it to your liking; default is 1.85in).
%         \spacewidth is width of area between left and right boxes.
%
%\newlength{\rcollength}\setlength{\rcollength}{0in}%
\newlength{\spacewidth}\setlength{\spacewidth}{20pt}
%
%\begin{tabular}[t]{@{}p{\textwidth-\rcollength-\spacewidth}@{}p{\spacewidth}@{}p{\rcollength}}%

{-- ------ ----- \hfill Mobile: +44 (0)---- ------ \\
-----, ---- \hfill{Web: \href{http://www.pauljwright.co.uk}{www.pauljwright.co.uk}} \\ 
---- --- \hfill{Email: paul@pauljwright.co.uk} \\ 
United Kingdom \hfill{Publication List: \href{https://ui.adsabs.harvard.edu/#search/q=orcid\%3A0000-0001-9021-611X&sort=date+desc}{SAO/NASA ADS}}}

%\hfill{GitHub: \href{http://www.github.com/pauljwright}{www.github.com/pauljwright}} }


%\end{tabular}

%%
%% In modern CV's, it seems like ``Objective'' is frowned upon. Instead,
%% incorporate it into a well-constructed cover letter. The ``More
%% information'' can go at the end of the CV, but it should not distract
%% from the section giving references available to contact.
%%
%
% \section{Objective}
%
% Placement in an academic position (i.e., faculty, postdoctoral, or
% research scientist) that allows for advanced research in distributed
% complex adaptive systems (i.e., modeling, analysis, design, and
% verification) with a particular focus on the control of engineered
% agents (e.g., for communications, control, software, electronics, and
% sustainability) and the analysis of biological phenomena (e.g.,
% self-organization, ecological rationality)
% \begin{innerlist}
% \item More information and auxiliary documents can be found at\\\url{http://www.tedpavlic.com/facjobsearch/}
% \end{innerlist}

%\section{Research Summary}

%My interests range from stellar to solar physics; my main interests lie in the heating of the solar atmosphere, including active regions and loops. I have expertise in the analysis of spectroscopic and narrowband Extreme Ultra-Violet (EUV) and X-ray data from the {\it Solar Dynamics Observatory} and {\it Hinode}, in addition to the hard X-ray (HXR) observations from {\it NuSTAR}'s heliophysics campaign. I am currently using a hydrodynamics code (Enthalpy-Based Thermal Evolution of Loop, EBTEL) to model light curves from coronal loops. I am particularly interested in the weak bremsstrahlung emission, with a view to understanding the requirements for future X-ray instrumentation to study the coronal heating problem.

%My research interests range from solar to stellar physics, with the majority of my research concentrating on the heating of the solar corona. I have expertise in the analysis of spectroscopic and narrowband Extreme Ultra-Violet (EUV) and X-ray data from the {\it Solar Dynamics Observatory} and {\it Hinode}, in addition to the hard X-ray (HXR) observations from {\it NuSTAR}'s heliophysics campaign. I am currently using a hydrodynamics code (Enthalpy-Based Thermal Evolution of Loops, EBTEL) to model light curves from coronal loops. In particular, I am interested in the weak bremsstrahlung emission, with a view to understanding the requirements for future X-ray instrumentation to study the coronal heating problem.


%My research interests are in the fields of solar and stellar flares, and my Ph.D. research has concentrated on one of the unsolved questions in Heliophysics -- the coronal heating problem. I have expertise of image and spectroscopic data analysis from EUV and X-ray data from the {\it Solar Dynamics Observatory} and {\it Hinode} in addition to expertise in analysing the hard X-ray data obtained during {\it NuSTAR}'s heliophysics campaign, and observations of stellar flares observed by {\it Kepler}. I am also currently using a hydrodynamics code (Enthalpy-Based Thermal Evolution of Loops, EBTEL) to model light curves from coronal loops. %%% --- 17th March

%My research interests are in solar and stellar physics, and my Ph.D. research has concentrated on one of the unsolved problems in Heliophysics---the coronal heating problem. During my Ph.D. I have gained expertise in numerous time-series analysis techniques and methods for recovering the differential emission measure (an ill-posed inverse problem) from a wide range of spectroscopic and narrowband data. I am a member of the {\it NuSTAR} Heliophysics working group and I led the analysis of the first solar flare observed by the {\it NuSTAR} hard X-ray {\it astrophysics} imaging spectrometer. I have also developed a stellar flare detection algorithm based on the observations obtained by the {\it Kepler} space telescope to determine the superflare rate of the Sun.


%My research interests range from solar to stellar physics with the majority of my Ph.D. research concentrating on the heating of the solar corona. I have expertise in the analysis of spectroscopic and narrowband Extreme Ultra-Violet (EUV) and X-ray data from the {\it Solar Dynamics Observatory} and {\it Hinode}, in addition to the hard X-ray (HXR) observations from {\it NuSTAR}'s heliophysics campaign, and white-light observations from {\it Kepler}. I am also currently using a hydrodynamics code (Enthalpy-Based Thermal Evolution of Loops, EBTEL) to model light curves from coronal loops.


%My interests range from stellar to solar physics; my main interests lie in the heating of the solar atmosphere, including active regions and loops. I have expertise in the analysis of Extreme Ultra-Violet (EUV) and X-ray data from the {\it SDO}, and {\it Hinode} satellites, in addition to the hard X-ray (HXR) imaging/spectroscopic observations from \textit{NuSTAR}'s heliophysics campaign. Furthermore, I am currently modelling coronal light-curves using the \texttt{EBTEL} (Enthalpy-Based Thermal Evolution of Loops) hydrodynamics code with particular interest in the weak bremmstrahlung components.

%My interests range from stellar to solar physics; my main interests lie in the heating of the solar atmosphere, including active regions and loops. I have expertise in analysis of data from \textit{Kepler}, \textit{SDO}/AIA, \textit{Hinode}/EIS, \textit{Hinode}/XRT, and \textit{NuSTAR}'s heliophysics observations. In addition, I am currently modelling coronal light-curves using the \texttt{EBTEL} (Enthalpy-Based Thermal Evolution of Loops) hydrodynamics code.

%%%% EDUCATION %%%%
\section{Education}

\href{http://www.gla.ac.uk/}{\textbf{University of Glasgow}},
Glasgow, UK \hfill {10/2014 -- 04/2019}
\begin{innerlist}
\item[] Ph.D.
        \href{}
             {Solar Physics}
        \begin{innerlist}
        \item[] Thesis Title: \emph{The Energetics of Small Solar Flares and Brightenings}
        \item[] Advisers: Dr Iain G. Hannah, Dr Alexander MacKinnon
        \end{innerlist}

\end{innerlist}

%\halfblankline

\href{http://www.soton.ac.uk/}{\textbf{University of Southampton}},
Southampton, UK \hfill {10/2010 -- 06/2014}
\begin{innerlist}
\item[] MPhys
        \href{http://www.phys.soton.ac.uk/programmes/f3fm-mphys-astronomy-year-abroad-4-yrs/}
             {Astrophysics with a year abroad}
        \begin{innerlist}
        	   \item[] First-class honours (1:1)
%        \end{innerlist}
%\end{innerlist}
%
%\href{https://www.cfa.harvard.edu/}{\textbf{Smithsonian Astrophysical Observatory}},
%Cambridge, MA, USA 
%\hfill {10/2013 -- 06/2014}
%\begin{innerlist}
%\item[] MPhys
%        \href{http://www.phys.soton.ac.uk/programmes/f3fm-mphys-astronomy-year-abroad-4-yrs/}
%             {Astrophysics with a year abroad}
%        \begin{innerlist}
        \item[] Thesis Title: \emph{The Superflare Rates of Solar-Like Stars}
        \item[] Advisers: Dr Steven H. Saar, Dr Jeremy J. Drake
        \end{innerlist}
\end{innerlist}

\section{Current Appointments}

\textbf{Postdoctoral Research Fellow},
            \href{http://www.stanford.edu}{Stanford University}
            \hfill {06/2019 -- }
\begin{innerlist}
    \item[] \href{http://hepl.stanford.edu/}{W. W. Hansen Experimental Physics Laboratory}
\end{innerlist}

%\halfblankline

\textbf{Mentor},
            \href{http://www.frontierdevelopmentlab.org/}{NASA Frontier Development Lab (FDL)}
            \hfill {06/2019 -- }
\begin{innerlist}
\item[] \href{http://seti.org/}{SETI Institute}/\href{}{NASA Ames Research Center, Mountain View, CA}
    \begin{innerlist}
    \item[] Project: {\it Super-resolution magnetograms}
\end{innerlist}            
\end{innerlist}            

\section{Selected Previous Appointments}
\textbf{Affiliate Staff Member},
            \href{http://www.gla.ac.uk/}{University of Glasgow}
            \hfill {10/2017 -- 04/2019}
\begin{innerlist}
\begin{innerlist}
  	\item{} Used the EBTEL hydrodynamics code to model light curves from coronal loops. The parameter space of these simulations will be constrained by observations obtained during the {\it NuSTAR} heliophysics campaign, and to test a variety of coronal analysis techniques.
\end{innerlist}
\end{innerlist}
%\halfblankline

\textbf{Researcher},
            \href{http://www.frontierdevelopmentlab.org/}{NASA Frontier Development Lab (FDL)}
            \hfill {06/2018 - 08/2018}
    	   \begin{innerlist} 
%    	\item[] \href{http://seti.org/}{SETI Institute}/\href{}{NASA Ames Research Center, Mountain View, CA}
  	    \begin{innerlist}
  	    \item{} Predicted MEGS-A Solar Spectral Irradiance (SSI) with median absolute relative uncertainties of less than $1.6\%$ per emission line using a Convolutional Neural Network (CNN) augmented with a Multi-Layer Perceptron (MLP).
  	\item{} Used a 1x1 CNN to improve the computational speed ($10^{3}\times$ increase) for differential emission measure (DEM) inversion. 
    	   \end{innerlist}
    	   \end{innerlist}

%\halfblankline
    
\textbf{Post-Graduate Research Assistant},
            \href{http://www.gla.ac.uk/}{University of Glasgow}
            \hfill {10/2014 -- 07/2017}
\begin{innerlist}
%    \item[] \href{http://www.astro.gla.ac.uk/}{SUPA School of Physics and Astronomy}
    \begin{innerlist}
    	    	\item{} Analysed observations of the Sun with {\it NuSTAR}, a telescope not designed for heliophysics. These observations are the most sensitive of their kind and have resulted in numerous, wide-ranging, highly-collaborative peer-reviewed publications.
    	    	    	\item{} Studied the temperature distribution of the solar atmosphere through the recovery of an ill-posed inverse problem (the differential emission measure, DEM) using techniques such as Tikhonov regularisation, Markov-chain Monte Carlo, Spline fitting, and Sparse Inversion (by Basis Pursuit).
    \end{innerlist}
\end{innerlist}

%\halfblankline

\textbf{Visiting Researcher}, \href{https://www.nasa.gov/goddard}{NASA Goddard Space Flight Center (GSFC)} 
\hfill {04/2016}
\begin{innerlist}\begin{innerlist}
    	\item{} Explored the possibility of implementing DEM maps in the \href{www.helioviewer.org}{Helioviewer} project, and their usefulness as an input for various established time-series analysis techniques.
    	\end{innerlist}\end{innerlist}
    	
%\halfblankline
    	
\textbf{Research Scholar},
\href{https://www.cfa.harvard.edu/}{Center for Astrophysics | Harvard \& Smithsonian} 
\hfill {10/2013 -- 06/2014}
\begin{innerlist}\begin{innerlist}
    	\item{} Designed and implemented a sophisticated stellar flare detection routine for long-cadence (30 mins) {\it Kepler} data obtained from a proprietary set of spectroscopically verified solar-type stars in three open clusters: this work had coverage by \href{http://www.sciencemag.org/news/2015/08/when-sun-s-next-superflare-due?rss=1}{Science} and \href{http://www.smithsonianmag.com/smart-news/when-next-solar-superflare-hit-earth-180956288/}{the Smithsonian}. % Magazine}.
    	\end{innerlist}\end{innerlist}
%
%\textbf{Summer Research Intern},
%\href{http://www.soton.ac.uk/}{University of Southampton} 
%\hfill {07/2013}

\section{Memberships}
     \href{https://www.nustar.caltech.edu/page/sun}{\textbf{\textit{NuSTAR} Heliophysics Working Group}}, Member \hfill {2015 -- present} \\
     \href{https://www.ras.org.uk/}{\textbf{Royal Astronomical Society}}, RAS Fellow \hfill {2014 -- present} \\
     \href{http://www.issibern.ch/}{\textbf{International Space Science Institute (ISSI)}}, Young Scientist Member \hfill {2015 -- 2018}
    \begin{innerlist}
     \item[] Member of Paola Testa's ISSI Team: \href{http://helio.cfa.harvard.edu/ISSI/welcome_pt.html}{\textit{New Diagnostics of Particle Acceleration in Solar Coronal Nanoflares from Chromospheric Observations and Modelling}}
    \end{innerlist}
% 
%\section{Technical Skills:} {\it Computing:} IDL (5+ years), Python (2+ years), PyTorch, R, Bash, $\LaTeX$, PyCharm, IRAF, git (GitHub, Gitlab), Microsoft Office, Adobe Creative Cloud, Linux/Unix, Mac OSX%, Microsoft Windows
%
%%\halfblankline
%
%{\it General:} Data Analysis, Data Visualisation, Interdisciplinary Collaboration, Public Speaking, Teaching, Writing (Technical \& Lay) 
%
% % Add a little space to nudge next ``Ref'd Journal Publications'' marginpar
% % down to make room for tall ``Submitted Journal Publications''
% % marginpar. If there are enough submitted journal publications, this
% % space will not be needed (and should be removed).
% \vspace{0.1in}

\newpage
\makeheading{Paul James Wright \hfill {List of Publications}} 

\section{Refereed Journal Publications}
\begin{bibenum}
	\item{} Marsh, A.~J., Smith, D.~M., Glesener, L. \textit{et al} 2017. {\it First NuSTAR Limits on Quiet Sun Hard X-Ray Transient Events}, \MYhref[magenta]{https://doi.org/10.3847/1538-4357/aa9122}{{\it ApJ}}, \href{https://ui.adsabs.harvard.edu/?#abs/2017ApJ...849..131M}{849, 131}
	\item{} Wang, J., Sim{\~o}es, P.~J.~A., Jeffrey, N.~L.~S. \textit{et al} 2017. {\it Observations of Reconnection Flows in a Flare on The Solar Disk}, \MYhref[magenta]{https://doi.org/10.3847/2041-8213/aa8904}{{\it ApJL}}, \href{http://adsabs.harvard.edu/abs/2017arXiv170808706W}{847, L1}
	\item{} {\bf Wright, P. J.}, Hannah, I.~G., Grefenstette, B. W., \textit{et al} 2017. {\it Microflare Heating of a Solar Active Region Observed with {\it NuSTAR}, {\it Hinode}/XRT, and {\it SDO}/AIA}, \MYhref[magenta]{https://doi.org/10.3847/1538-4357/aa7a59}{{\it ApJ}}, \href{http://adsabs.harvard.edu/abs/2017ApJ...844..132W}{844, 132}
	\item{} Kuhar, M., Krucker, S., Hannah, I. G., \textit{et al} 2017. {\it Evidence of Significant Energy Input in the Late Phase of a Solar Flare from NuSTAR X-ray Observations}, \MYhref[magenta]{http://doi:10.3847/1538-4357/835/1/6}{{\it ApJ}}, \href{http://adsabs.harvard.edu/abs/2017ApJ...835....6K}{835, 6}
	\item{} Galvez, R., Fouhey, D. F., Jin, M., \textit{et al} 2019. {\it A Machine Learning Dataset Prepared From the NASA Solar Dynamics Observatory Mission}, \MYhref[magenta]{http://doi:10.3847/1538-4357/835/1/6}{{\it ApJS}}, \href{https://ui.adsabs.harvard.edu/#abs/arXiv:1903.04538}{(accepted)}	
	
	
\end{bibenum}

\section{Book Chapters}
\begin{bibenum}
 	\item{} {\bf Wright, P. J.}, Cheung, M.~C.~M., Thomas, R., \textit{et al} 2018 {\it DeepEM: A Deep Learning Approach to DEM Inversion}. In M. Bobra \& J. Mason, eds., {\it Machine Learning, Statistics, and Data Mining for Heliophysics}, \href{https://helioml.github.io/HelioML/}{Chapter 4}
\end{bibenum}

\section{First Author Publications in Preparation (Working Titles)}
\begin{bibenum}
 	\item{} {\bf Wright, P. J.}, Galvez, R., \textit{et al} 2019. {\it DeepEM: A Deep Learning Approach to DEM Inversion}
% 	\item{} {\bf Wright, P. J.}, MacKinnon, A., Hannah, I.~G., and Sim{\~o}es, P.~J.~A. 2019. {\it Local Intermittency Measure: The Application to Active Region Light Curves}
	\item{} {\bf Wright, P. J.}, Hannah, I.~G., Viall, N.~M., \textit{et al} 2019. {\it The Thermal Time Evolution of Active Regions Determined by SDO/AIA}
	\item{} {\bf Wright, P. J.}, Saar, S.~H., Meibom, S., \textit{et al} 2019. {\it  The Age-Dependent Superflare Rates of G-Type Dwarfs In Three {\it Kepler} Clusters}
%	\item{} {\bf Wright, P. J.}, Saar, S.~H., Meibom, S., \textit{et al} 2019. {\it An Extension of The Age-Dependent Superflare Rates to F- and K-Type Dwarfs}
\end{bibenum}

\end{document}
