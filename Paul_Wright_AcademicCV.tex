%%%%%%%%%%%%%%%%%%%%%%%%%%%%%%%%%%%%%%%%%%%%%%%%%%%%%%%%%%%%%%%%%%%%%%%%
%%%%%%%%%%%%%%%%%%%%%% Simple LaTeX CV Template %%%%%%%%%%%%%%%%%%%%%%%%
%%%%%%%%%%%%%%%%%%%%%%%%%%%%%%%%%%%%%%%%%%%%%%%%%%%%%%%%%%%%%%%%%%%%%%%%

%%%%%%%%%%%%%%%%%%%%%%%%%%%%%%%%%%%%%%%%%%%%%%%%%%%%%%%%%%%%%%%%%%%%%%%%
%% NOTE: If you find that it says                                     %%
%%                                                                    %%
%%                           1 of ??                                  %%
%%                                                                    %%
%% at the bottom of your first page, this means that the AUX file     %%
%% was not available when you ran LaTeX on this source. Simply RERUN  %%
%% LaTeX to get the ``??'' replaced with the number of the last page  %%
%% of the document. The AUX file will be generated on the first run   %%
%% of LaTeX and used on the second run to fill in all of the          %%
%% references.                                                        %%
%%%%%%%%%%%%%%%%%%%%%%%%%%%%%%%%%%%%%%%%%%%%%%%%%%%%%%%%%%%%%%%%%%%%%%%%

%%%%%%%%%%%%%%%%%%%%%%%%%%%% Document Setup %%%%%%%%%%%%%%%%%%%%%%%%%%%%

% Don't like 10pt? Try 11pt or 12pt
\documentclass[11pt]{article}
\RequirePackage[T1]{fontenc}

% LaTeX will typeset using Computer Modern Roman, which a lot of
% non-mathematicians and non-engineers won't like. Also, a few PDF
% viewers may not render CMR very well. Instead, Times New Roman can
% be used. That's what this package does.
\usepackage{times}

% The automated optical recognition software used to digitize resume
% information works best with fonts that do not have serifs. This
% command uses a sans serif font throughout. Uncomment both lines (or at
% least the second) to restore a Roman font (i.e., a font with serifs).
% (NOTE: This requires the times package above)
%\renewcommand{\familydefault}{\sfdefault}

% This is a helpful package that puts math inside length specifications
\usepackage{calc}

% This package helps LaTeX auto-hyphenate hyphenated words if you use
% special hyphens. For example, bio\-/mimicry will properly hyphenate
% ``mimicry'' if necessary.
\usepackage[shortcuts]{extdash}

% Layout: Puts the section titles on left side of page
\reversemarginpar

%
%         PAPER SIZE, PAGE NUMBER, AND DOCUMENT LAYOUT NOTES:
%
% The next \usepackage line changes the layout for CV style section
% headings as marginal notes. It also sets up the paper size as either
% letter or A4. By default, letter was used. If A4 paper is desired,
% comment out the letterpaper lines and uncomment the a4paper lines.
%
% As you can see, the margin widths and section title widths can be
% easily adjusted.
%
% ALSO: Notice that the includefoot option can be commented OUT in order
% to put the PAGE NUMBER *IN* the bottom margin. This will make the
% effective text area larger.
%
% IF YOU WISH TO REMOVE THE ``of LASTPAGE'' next to each page number,
% see the note about the +LP and -LP lines below. Comment out the +LP
% and uncomment the -LP.
%
% IF YOU WISH TO REMOVE PAGE NUMBERS, be sure that the includefoot line
% is uncommented and ALSO uncomment the \pagestyle{empty} a few lines
% below.
%

%% Use these lines for letter-sized paper
\usepackage[paper=a4paper,
            %includefoot, % Uncomment to put page number above margin
            marginparwidth=1.2in,     % Length of section titles
            marginparsep=.05in,       % Space between titles and text
            margin=0.6in,               % 1 inch margins
            includemp]{geometry}

%% Use these lines for A4-sized paper
%\usepackage[paper=a4paper,
%            %includefoot, % Uncomment to put page number above margin
%            marginparwidth=30.5mm,    % Length of section titlesa
%            marginparsep=1.5mm,       % Space between titles and text
%            margin=25mm,              % 25mm margins
%            includemp]{geometry}

%% More layout: Get rid of indenting throughout entire document
\setlength{\parindent}{0in}

% Provides special list environments and macros to create new ones
\usepackage[shortlabels]{enumitem}

% Simpler bibsections for CV sections
% (thanks to natbib for inspiration)
%
% * For lists of references with hanging indents and no numbers:
%
%   \begin{bibsection}
%       \item ...
%   \end{bibsection}
%
% * For numbered lists of references (with hanging indents):
%
%   \begin{bibenum}
%       \item ...
%   \end{bibenum}
%
%   Note that bibenum numbers continuously throughout. To reset the
%   counter, use
%
%   \restartlist{bibenum}
%
%   at the place where you want the numbering to reset.

\makeatletter
\newlength{\bibhang}
\setlength{\bibhang}{1em}
\newlength{\bibsep}
 {\@listi \global\bibsep\itemsep \global\advance\bibsep by\parsep}
\newlist{bibsection}{itemize}{3}
\setlist[bibsection]{label=,leftmargin=\bibhang,%
        itemindent=-\bibhang,
        itemsep=\bibsep,parsep=\z@,partopsep=0pt,
        topsep=0pt}
\newlist{bibenum}{enumerate}{3}
\setlist[bibenum]{label=[\arabic*],resume,leftmargin={\bibhang+\widthof{[999]}},%
        itemindent=-\bibhang,
        itemsep=\bibsep,parsep=\z@,partopsep=0pt,
        topsep=0pt}
\let\oldendbibenum\endbibenum
\def\endbibenum{\oldendbibenum\vspace{-.6\baselineskip}}
\let\oldendbibsection\endbibsection
\def\endbibsection{\oldendbibsection\vspace{-.6\baselineskip}}
\makeatother

%% Reference the last page in the page number
%
% NOTE: comment the +LP line and uncomment the -LP line to have page
%       numbers without the ``of ##'' last page reference)
%
% NOTE: uncomment the \pagestyle{empty} line to get rid of all page
%       numbers (make sure includefoot is commented out above)
%
\usepackage{fancyhdr,lastpage}
\pagestyle{fancy}
%\pagestyle{empty}      % Uncomment this to get rid of page numbers
\fancyhf{}\renewcommand{\headrulewidth}{0pt}
\fancyfootoffset{\marginparsep+\marginparwidth}
\newlength{\footpageshift}
\setlength{\footpageshift}
          {0.5\textwidth+0.5\marginparsep+0.5\marginparwidth-2in}
\lfoot{\hspace{\footpageshift}%
       \parbox{4in}{\, \hfill %
                    \arabic{page} of \protect\pageref*{LastPage} % +LP
%                    \arabic{page}                               % -LP
                    \hfill \,}}

% Finally, give us PDF bookmarks
\usepackage{color,hyperref}
\definecolor{darkblue}{rgb}{0.0,0.0,0.3}
\hypersetup{colorlinks,breaklinks,
            linkcolor=darkblue,urlcolor=darkblue,
            anchorcolor=darkblue,citecolor=darkblue}
\newcommand{\MYhref}[3][blue]{\href{#2}{\color{#1}{#3}}}%
%%%%%%%%%%%%%%%%%%%%%%%% End Document Setup %%%%%%%%%%%%%%%%%%%%%%%%%%%%


%%%%%%%%%%%%%%%%%%%%%%%%%%% Helper Commands %%%%%%%%%%%%%%%%%%%%%%%%%%%%

%%% HEADING AT TOP OF CURRICULUM VITAE

% The title (name) with a horizontal rule under it
% (optional argument typesets an object right-justified across from name
%  as well)
%
% Usage: \makeheading{name}
%        OR
%        \makeheading[right_object]{name}
%
% Place at top of document. It should be the first thing.
% If ``right_object'' is provided in the square-braced optional
% argument, it will be right justified on the same line as ``name'' at
% the top of the CV. For example:
%
%       \makeheading[\emph{Curriculum vitae}]{Your Name}
%
% will put an emphasized ``Curriculum vitae'' at the top of the document
% as a title. Likewise, a picture could be included:
%
%   \makeheading[{\includegraphics[height=1.5in]{my_picture}}]{Your Name}
%
% the picture will be flush right across from the name. For this example
% to work, make sure the extra set of curly braces is included. Also
% makes ure that \usepackage{graphicx} is somewhere in the preamble.
\newcommand{\makeheading}[2][]%
        {\hspace*{-\marginparsep minus \marginparwidth}%
         \begin{minipage}[t]{\textwidth+\marginparwidth+\marginparsep}%
             {\large \bfseries #2 \hfill #1}\\[-0.15\baselineskip]%
                 \rule{\columnwidth}{1pt}%
         \end{minipage}}

%%% SECTION HEADINGS

% The section headings. Flush left in small caps down pseudo-margin.
%
% Usage: \section{section name}
\renewcommand{\section}[1]{\pagebreak[3]%
    \vspace{1.3\baselineskip}%
    \phantomsection\addcontentsline{toc}{section}{#1}%
    \noindent\llap{\scshape\smash{\parbox[t]{\marginparwidth}{\hyphenpenalty=10000\raggedright #1}}}%
    \vspace{-\baselineskip}\par}

%%% LISTS

% This macro alters a list by removing some of the space that follows the list
% (is used by lists below)
\newcommand*\fixendlist[1]{%
    \expandafter\let\csname preFixEndListend#1\expandafter\endcsname\csname end#1\endcsname
    \expandafter\def\csname end#1\endcsname{\csname preFixEndListend#1\endcsname\vspace{-0.6\baselineskip}}}

% These macros help ensure that items in outer-type lists do not get
% separated from the next line by a page break
% (they are used by lists below)
\let\originalItem\item
\newcommand*\fixouterlist[1]{%
    \expandafter\let\csname preFixOuterList#1\expandafter\endcsname\csname #1\endcsname
    \expandafter\def\csname #1\endcsname{\let\oldItem\item\def\item{\pagebreak[2]\oldItem}\csname preFixOuterList#1\endcsname}
    \expandafter\let\csname preFixOuterListend#1\expandafter\endcsname\csname end#1\endcsname
    \expandafter\def\csname end#1\endcsname{\let\item\oldItem\csname preFixOuterListend#1\endcsname}}
\newcommand*\fixinnerlist[1]{%
    \expandafter\let\csname preFixInnerList#1\expandafter\endcsname\csname #1\endcsname
    \expandafter\def\csname #1\endcsname{\let\oldItem\item\let\item\originalItem\csname preFixInnerList#1\endcsname}
    \expandafter\let\csname preFixInnerListend#1\expandafter\endcsname\csname end#1\endcsname
    \expandafter\def\csname end#1\endcsname{\csname preFixInnerListend#1\endcsname\let\item\oldItem}}

% An itemize-style list with lots of space between items
%
% Usage:
%   \begin{outerlist}
%       \item ...    % (or \item[] for no bullet)
%   \end{outerlist}
\newlist{outerlist}{itemize}{3}
    \setlist[outerlist]{label=\enskip\textbullet,leftmargin=*}
    \fixendlist{outerlist}
    \fixouterlist{outerlist}

% An environment IDENTICAL to outerlist that has better pre-list spacing
% when used as the first thing in a \section
%
% Usage:
%   \begin{lonelist}
%       \item ...    % (or \item[] for no bullet)
%   \end{lonelist}
\newlist{lonelist}{itemize}{3}
    \setlist[lonelist]{label=\enskip\textbullet,leftmargin=*,partopsep=0pt,topsep=0pt}
    \fixendlist{lonelist}
    \fixouterlist{lonelist}

% An itemize-style list with little space between items
%
% Usage:
%   \begin{innerlist}
%       \item ...    % (or \item[] for no bullet)
%   \end{innerlist}
\newlist{innerlist}{itemize}{3}
    \setlist[innerlist]{label=\enskip\textbullet,leftmargin=*,parsep=0pt,itemsep=0pt,topsep=0pt,partopsep=0pt}
    \fixinnerlist{innerlist}

% An environment IDENTICAL to innerlist that has better pre-list spacing
% when used as the first thing in a \section
%
% Usage:
%   \begin{loneinnerlist}
%       \item ...    % (or \item[] for no bullet)
%   \end{loneinnerlist}
\newlist{loneinnerlist}{itemize}{3}
    \setlist[loneinnerlist]{label=\enskip\textbullet,leftmargin=*,parsep=0pt,itemsep=0pt,topsep=0pt,partopsep=0pt}
    \fixendlist{loneinnerlist}
    \fixinnerlist{loneinnerlist}

%%% EXTRA SPACE

% To add some paragraph space between lines.
% This also tells LaTeX to preferably break a page on one of these gaps
% if there is a needed pagebreak nearby.
\newcommand{\blankline}{\quad\pagebreak[3]}
\newcommand{\halfblankline}{\quad\vspace{-0.5\baselineskip}\pagebreak[3]}

%%% FORMATTING MACROS

% Provides a linked \doi{#1} that links doi:#1 to http://dx.doi.org/#1
\usepackage{doi}
% To change the text before the DOI, adjust this command
%\renewcommand\doitext{doi:}

% Provides a linked \url{#1} that doesn't require escape characters
\usepackage{url}

% You can adjust the style \url{} uses here:
% (options are: same, rm, sf, tt; defaults to tt)
\urlstyle{same}

% For \email{ADDRESS}, links ADDRESS to the url mailto:ADDRESS
% (uncomment to typeset the e\-/mail address in typewriter font;
%  otherwise, will be typeset in the \urlstyle above)
%\DeclareUrlCommand\emaillink{\urlstyle{tt}}
\providecommand*\emaillink[1]{\nolinkurl{#1}}
\providecommand*\email[1]{\href{mailto:#1}{\emaillink{#1}}}

\providecommand\BibTeX{{B\kern-.05em{\sc i\kern-.025em b}\kern-.08em \TeX}}
\providecommand\Matlab{\textsc{Matlab}}

% Custom hyphenation rules for words that LaTeX has trouble with
\hyphenation{bio-mim-ic-ry bio-in-spi-ra-tion re-us-a-ble pro-vid-er Media-Wiki}

%%%%%%%%%%%%%%%%%%%%%%%% End Helper Commands %%%%%%%%%%%%%%%%%%%%%%%%%%%

%%%%%%%%%%%%%%%%%%%%%%%%% Begin CV Document %%%%%%%%%%%%%%%%%%%%%%%%%%%%

\begin{document}
\makeheading{Paul James Wright}

\section{Contact Information}

% NOTE: Mind where the & separators and \\ breaks are in the following
%       table. Table is one row made up of three parboxes. The left
%       parbox has address info, the middle parbox has a vertical bar,
%       and the right parbox has phone and electronic contact
%       information.
%
% MACROS: \rcollength is the width of the right column of the table
%             (adjust it to your liking; default is 1.85in).
%         \spacewidth is width of area between left and right boxes.
%
%\newlength{\rcollength}\setlength{\rcollength}{0in}%
\newlength{\spacewidth}\setlength{\spacewidth}{20pt}
%
%\begin{tabular}[t]{@{}p{\textwidth-\rcollength-\spacewidth}@{}p{\spacewidth}@{}p{\rcollength}}%

{Rm~614, Kelvin Building \hfill Work: +44 (0)14133 08855 \\
University of Glasgow \hfill{Web: \href{http://www.pauljwright.co.uk}{www.pauljwright.co.uk}} \\ 
Glasgow, G12 8QQ \hfill{Email: paul.wright@glasgow.ac.uk} \\ 
United Kingdom \hfill{Publication List: \href{https://ui.adsabs.harvard.edu/#search/q=orcid\%3A0000-0001-9021-611X&sort=date+desc}{SAO/NASA ADS}}}

%\hfill{GitHub: \href{http://www.github.com/pauljwright}{www.github.com/pauljwright}} }


%\end{tabular}

%%
%% In modern CV's, it seems like ``Objective'' is frowned upon. Instead,
%% incorporate it into a well-constructed cover letter. The ``More
%% information'' can go at the end of the CV, but it should not distract
%% from the section giving references available to contact.
%%
%
% \section{Objective}
%
% Placement in an academic position (i.e., faculty, postdoctoral, or
% research scientist) that allows for advanced research in distributed
% complex adaptive systems (i.e., modeling, analysis, design, and
% verification) with a particular focus on the control of engineered
% agents (e.g., for communications, control, software, electronics, and
% sustainability) and the analysis of biological phenomena (e.g.,
% self-organization, ecological rationality)
% \begin{innerlist}
% \item More information and auxiliary documents can be found at\\\url{http://www.tedpavlic.com/facjobsearch/}
% \end{innerlist}

\section{Research Summary}

%My interests range from stellar to solar physics; my main interests lie in the heating of the solar atmosphere, including active regions and loops. I have expertise in the analysis of spectroscopic and narrowband Extreme Ultra-Violet (EUV) and X-ray data from the {\it Solar Dynamics Observatory} and {\it Hinode}, in addition to the hard X-ray (HXR) observations from {\it NuSTAR}'s heliophysics campaign. I am currently using a hydrodynamics code (Enthalpy-Based Thermal Evolution of Loop, EBTEL) to model light curves from coronal loops. I am particularly interested in the weak bremsstrahlung emission, with a view to understanding the requirements for future X-ray instrumentation to study the coronal heating problem.

%My research interests range from solar to stellar physics, with the majority of my research concentrating on the heating of the solar corona. I have expertise in the analysis of spectroscopic and narrowband Extreme Ultra-Violet (EUV) and X-ray data from the {\it Solar Dynamics Observatory} and {\it Hinode}, in addition to the hard X-ray (HXR) observations from {\it NuSTAR}'s heliophysics campaign. I am currently using a hydrodynamics code (Enthalpy-Based Thermal Evolution of Loops, EBTEL) to model light curves from coronal loops. In particular, I am interested in the weak bremsstrahlung emission, with a view to understanding the requirements for future X-ray instrumentation to study the coronal heating problem.


%My research interests are in the fields of solar and stellar flares, and my Ph.D. research has concentrated on one of the unsolved questions in Heliophysics -- the coronal heating problem. I have expertise of image and spectroscopic data analysis from EUV and X-ray data from the {\it Solar Dynamics Observatory} and {\it Hinode} in addition to expertise in analysing the hard X-ray data obtained during {\it NuSTAR}'s heliophysics campaign, and observations of stellar flares observed by {\it Kepler}. I am also currently using a hydrodynamics code (Enthalpy-Based Thermal Evolution of Loops, EBTEL) to model light curves from coronal loops. %%% --- 17th March

My research interests are in solar and stellar physics, and my Ph.D. research has concentrated on one of the unsolved problems in Heliophysics -- the coronal heating problem. During my Ph.D. I have gained expertise in numerous time-series analysis techniques and methods for recovering the differential emission measure (an ill-posed inverse problem) from a wide range of spectroscopic and narrowband data. I am also a member of the {\it NuSTAR} Heliophysics working group and I led the analysis of the first solar flare observed by the {\it NuSTAR} hard X-ray {\it astrophysics} imaging spectrometer. I have also developed a stellar flare detection algorithm based on the observations obtained by the {\it Kepler} space telescope to determine the superflare rate of the Sun.


%My research interests range from solar to stellar physics with the majority of my Ph.D. research concentrating on the heating of the solar corona. I have expertise in the analysis of spectroscopic and narrowband Extreme Ultra-Violet (EUV) and X-ray data from the {\it Solar Dynamics Observatory} and {\it Hinode}, in addition to the hard X-ray (HXR) observations from {\it NuSTAR}'s heliophysics campaign, and white-light observations from {\it Kepler}. I am also currently using a hydrodynamics code (Enthalpy-Based Thermal Evolution of Loops, EBTEL) to model light curves from coronal loops.


%My interests range from stellar to solar physics; my main interests lie in the heating of the solar atmosphere, including active regions and loops. I have expertise in the analysis of Extreme Ultra-Violet (EUV) and X-ray data from the {\it SDO}, and {\it Hinode} satellites, in addition to the hard X-ray (HXR) imaging/spectroscopic observations from \textit{NuSTAR}'s heliophysics campaign. Furthermore, I am currently modelling coronal light-curves using the \texttt{EBTEL} (Enthalpy-Based Thermal Evolution of Loops) hydrodynamics code with particular interest in the weak bremmstrahlung components.

%My interests range from stellar to solar physics; my main interests lie in the heating of the solar atmosphere, including active regions and loops. I have expertise in analysis of data from \textit{Kepler}, \textit{SDO}/AIA, \textit{Hinode}/EIS, \textit{Hinode}/XRT, and \textit{NuSTAR}'s heliophysics observations. In addition, I am currently modelling coronal light-curves using the \texttt{EBTEL} (Enthalpy-Based Thermal Evolution of Loops) hydrodynamics code.

%%%% EDUCATION %%%%
\section{Education}

\href{http://www.gla.ac.uk/}{\textbf{University of Glasgow}},
Glasgow, UK \hfill {2014 -- present}
\begin{innerlist}
\item[] Ph.D.
        \href{}
             {Solar Physics}
        \begin{innerlist}
        \item[] Thesis Topic: \emph{The Energetics of Small Flares and Brightenings}
        \item[] Advisers:
              \href{}
                   {Dr Iain G. Hannah, Dr Alexander MacKinnon}
        \end{innerlist}

\end{innerlist}

\halfblankline

\href{http://www.soton.ac.uk/}{\textbf{University of Southampton}},
Southampton, UK \hfill {2010 -- 2014}
\begin{innerlist}

\item[] MPhys
        \href{http://www.phys.soton.ac.uk/programmes/f3fm-mphys-astronomy-year-abroad-4-yrs/}
             {Astrophysics with a year abroad}
        \begin{innerlist}
        \item[] First-class honours (1:1)
         \item[] Adviser:
              \href{}
                   {Professor Malcolm Coe}
        \end{innerlist}

\end{innerlist}

\halfblankline

\href{https://www.cfa.harvard.edu/}{\textbf{Harvard University/Harvard-Smithsonian CfA}},
Cambridge, MA, USA \hfill {2013 -- 2014}
\begin{innerlist}

\item[] MPhys
        \href{http://www.phys.soton.ac.uk/programmes/f3fm-mphys-astronomy-year-abroad-4-yrs/}
             {Astrophysics with a year abroad}
        \begin{innerlist}
        \item[] Thesis Topic: \emph{The Superflare Rates of Solar-Like Stars}
        \item[] Advisers:
              \href{}
                   {Dr Steven H. Saar, Dr Jeremy J. Drake}
        \end{innerlist}

\end{innerlist}

\section{Current Academic Appointments}

\textbf{Researcher},
            \href{http://www.frontierdevelopmentlab.org/}{NASA Frontier Development Lab (FDL)}
            \hfill {2018}
\begin{innerlist}

    \item[] \href{http://seti.org/}{SETI Institute}/\href{}{NASA Ames Research Center, Mountain View, CA}
    \begin{innerlist}
        	\item[] Project: \emph{Predicting Solar Spectral Irradiance from {\it SDO}/AIA Observations}
  	\item{} An 8-week applied Artificial Intelligence accelerator established to tackle knowledge gaps useful to NASA's science and exploration goals, and humanity. 
  	\item{} Implemented Deep Learning algorithms (Convolutional Neural Networks; CNNs) such as U-Net, AlexNet and ResNet to predict disk-integrated Solar Spectral Irradiance (SSI) observed by {\it SDO}/EVE (MEGS-A) from high-resolution {\it SDO}/AIA images which share a common latent space. 
  	\item{} Predicted MEGS-A SSI with average relative discrepancies of less than $3\%$ using a Residual Neural Network (ResNet) augmented with a Multi-Layer Perceptron (MLP).%\$20 Million on a new instrument.
  	%\item{} from four years of observations (0.7Tb per day?), in order 
  	\item{} Used a 1x1 CNN (equivalent to an MLP) to improve the computational speed for differential emission measure (DEM) inversion. Further improvement to the resulting DEMs were obtained by training a CNN to correct the DEMs to minimise the residual between observed and synthesized SSI.
  	\item{} Received the NASA Frontier Development Lab ``Contribution to Science'' award.
  	 \end{innerlist}

\end{innerlist}

\halfblankline

\textbf{Affiliate Staff Member},
            \href{http://www.gla.ac.uk/}{University of Glasgow}
            \hfill {2017 -- present}
\begin{innerlist}

    \item[] \href{http://astro.gla.ac.uk/}{SUPA School of Physics and Astronomy}
    \begin{innerlist}
  	\item{} Using the EBTEL hydrodynamics code to model light curves from coronal loops. The parameter space of these simulations will be constrained by observations obtained during the {\it NuSTAR} heliophysics campaign, and these simulations will be used to test a variety of analysis techniques.
    \end{innerlist}

\end{innerlist}
%
\newpage
\makeheading{Paul James Wright} 

\section{Previous Academic Appointments}
 
%\section{Academic Appointments}

 \textbf{Post-Graduate Research Assistant},
            \href{http://www.gla.ac.uk/}{University of Glasgow}
            \hfill {2014 -- 2017}
\begin{innerlist}

    \item[] \href{http://www.astro.gla.ac.uk/}{SUPA School of Physics and Astronomy}
    \begin{innerlist}
    	\item[] Project: \emph{The Energetics of Small Flares and Brightenings}
    	\item{} Analysed observations of the Sun with {\it NuSTAR}, a telescope not designed for heliophysics. These observations are the most sensitive of their kind and have resulted in numerous, wide-ranging, highly-collaborative peer-reviewed publications.
    	\item{} Analysed 20 million non-flaring coronal time-series in pursuit of signatures of the coronal heating mechanism. Techniques included time-lag analysis (cross-correlation), Fourier analysis, wavelet analysis, and local intermittency measure (LIM).
    	\item{} Studied the temperature distribution of the solar atmosphere through the recovery of an ill-posed inverse problem (the differential emission measure, DEM) using techniques such as Tikhonov regularisation, Markov-chain Monte Carlo, Spline fitting, and Sparse Inversion (by Basis Pursuit).
         \item{} The press-release image produced from the {\it NuSTAR} observations obtained for \href{https://doi.org/10.3847/1538-4357/aa7a59}{Wright {\it et al.} 2017} was published by numerous news outlets and is one of the five iconic images from \href{https://www.nasa.gov/feature/jpl/nustars-first-five-years-in-space}{{\it NuSTAR}'s first five years in space}.
    \end{innerlist}
        	\item[] Primary Collaborators: {\it Dr Iain Hannah, Dr Alexander MacKinnon, Dr Hugh Hudson, \\ Dr Paulo Sim{\~o}es}
\end{innerlist}
 
%\newpage
%\makeheading{Paul James Wright} 
%
\halfblankline

\textbf{Visiting Researcher},
	\href{https://www.nasa.gov/goddard}{NASA Goddard Space Flight Center (GSFC)} \hfill {2016}
\begin{innerlist}

    \item[] \href{https://science.gsfc.nasa.gov/heliophysics/}{Heliophysics Science Division}
    \begin{innerlist}
    	\item{} Explored the possibility of implementing DEM maps in the \href{www.helioviewer.org}{Helioviewer} project, and their usefulness as an input for various established analysis techniques.
    \end{innerlist}
        	\item[] Collaborators: {\it Dr Nicholeen Viall, Dr Jack Ireland}

\end{innerlist}

\halfblankline

 \textbf{Research Scholar},
	\href{https://www.cfa.harvard.edu/}{Harvard-Smithsonian Center for Astrophysics (CfA)} \hfill {2013 -- 2014}
\begin{innerlist}

    \item[] \href{https://www.cfa.harvard.edu/research/hea/sun}{Solar and Stellar X-ray Group}
    \begin{innerlist}
    	\item{} Designed and implemented a sophisticated stellar flare detection routine for long-cadence (30 mins) {\it Kepler} data obtained from a proprietary set of spectroscopically verified solar-type stars in three open clusters.\item{} A preliminary report on this work had coverage by \href{http://www.sciencemag.org/news/2015/08/when-sun-s-next-superflare-due?rss=1}{Science} and \href{http://www.smithsonianmag.com/smart-news/when-next-solar-superflare-hit-earth-180956288/}{the Smithsonian Magazine}.%, and \href{http://www.dailymail.co.uk/sciencetech/article-3198285/Is-Earth-hit-SUPERFLARE-Scientists-calculate-massive-solar-outburst-due.html}{The Daily Mail}.
    \end{innerlist}    	
    	\item[] Collaborators: {\it Dr Steven Saar, Dr S{\o}ren Meibom, Dr Jeremy Drake, Dr Jos{\'e} D. do Nascimento Jr, Dr Vinay Kashyap}
\end{innerlist}
\halfblankline

 \textbf{Summer Research Intern},
	\href{http://www.soton.ac.uk/}{University of Southampton} \hfill {2013}
\begin{innerlist}

    \item[] \href{http://www.astro.soton.ac.uk/}{Astronomy Group}
    \begin{innerlist}
    	\item{} Investigated the presence of double blue straggler sequences in globular clusters using Hubble Space Telescope (ACS, WFPC2) data.
    \end{innerlist}    	
    	\item[] Collaborators: {\it Dr Andrea Dieball}

\end{innerlist}

 
%
% % Add a little space to nudge next ``Ref'd Journal Publications'' marginpar
% % down to make room for tall ``Submitted Journal Publications''
% % marginpar. If there are enough submitted journal publications, this
% % space will not be needed (and should be removed).
% \vspace{0.1in}

\section{Refereed Journal Publications}

\begin{bibenum}
	\item{} Marsh, A.~J., Smith, D.~M., Glesener, L. \textit{et al} 2017. {\it First NuSTAR Limits on Quiet Sun Hard X-Ray Transient Events}, \MYhref[magenta]{https://doi.org/10.3847/1538-4357/aa9122}{{\it ApJ}}, \href{https://ui.adsabs.harvard.edu/?#abs/2017ApJ...849..131M}{849, 131}
	
	\item{} Wang, J., Sim{\~o}es, P.~J.~A., Jeffrey, N.~L.~S. \textit{et al} 2017. {\it Observations of Reconnection Flows in a Flare on The Solar Disk}, \MYhref[magenta]{https://doi.org/10.3847/2041-8213/aa8904}{{\it ApJL}}, \href{http://adsabs.harvard.edu/abs/2017arXiv170808706W}{847, L1}

	\item{} {\bf Wright, P. J.}, Hannah, I.~G., Grefenstette, B. W., \textit{et al} 2017. {\it Microflare Heating of a Solar Active Region Observed with {\it NuSTAR}, {\it Hinode}/XRT, and {\it SDO}/AIA}, \MYhref[magenta]{https://doi.org/10.3847/1538-4357/aa7a59}{{\it ApJ}}, \href{http://adsabs.harvard.edu/abs/2017ApJ...844..132W}{844, 132}

	\item{} Kuhar, M., Krucker, S., Hannah, I. G., \textit{et al} 2017. {\it Evidence of Significant Energy Input in the Late Phase of a Solar Flare from NuSTAR X-ray Observations}, \MYhref[magenta]{http://doi:10.3847/1538-4357/835/1/6}{{\it ApJ}}, \href{http://adsabs.harvard.edu/abs/2017ApJ...835....6K}{835, 6}

\end{bibenum}

%	\item{} Hannah, I. G.; {\bf Wright, P. J.}, Grefenstette, B. \textit{et al} 2015, American Geophysical Union, Fall Meeting, abstract \#SH13B-2442
%
%	\item{} Saar, S., Meibom, S., 
%\& {\bf Wright, P. J.}\ 2015, IAU General Assembly, 22, 58518 
%
%	\item{} Saar, S.~H., {\bf Wright, P. J.}, Meibom, S., Kashyap, V., 
%\& Drake, J.~J.\ 2014, American Astronomical Society Meeting Abstracts \#224, 224, \#123.43 
%
%	\item{} \textbf{Wright, P.~J.}, Saar, 
%S.~H., Meibom, S., Kashyap, V., 
%\& Drake, J.~J.\ 2014, American Astronomical Society Meeting Abstracts \#223, 223, \#151.08 
%
%	\item{}  Gorosabel, J., et al. (17 coauthors including {\bf Wright, P.})\ 2012, GRB 120327A: IAC80 I-Band Observations, GRB Coordinates Network, GCN Circular 13130.% \textbf{citations: 1}
%
%	\item{} Walker, C., et al. (16 coauthors including {\bf Wright, P.})\ 2012, GRB120326A: IAC80 \\ Observations, GRB Coordinates Network, GCN Circular 13112. %\textbf{citations: 4}


% Add a little space to nudge next ``Conference Publications'' marginpar
% down to make room for tall ``Submitted Conference Publications''
% marginpar. If there are enough submitted journal publications, this
% space will not be needed (and should be removed).
%\vspace{0.1in}

%\section{Conference Publications}

%\section{Conference Talks}

%\section{Conference Posters}

%\section{Invited Talks}

%\section{Book Chapters}

%\section{Other Publications}

%\section{Books in Preparation}


\newpage
\makeheading{Paul James Wright} 


\section{First Author Publications in Preparation (Working Titles)}

 \begin{bibenum}
 	%\item{} {\bf Wright, P. J.}, Galvez, R., \textit{et al} 2019. {\it DeepEM: A Deep Learning Approach to DEM Inversion}
 	\item{} {\bf Wright, P. J.}, MacKinnon, A., Hannah, I.~G., and Sim{\~o}es, P.~J.~A. 2019. {\it Local Intermittency Measure: The Application to Active Region Light Curves}
	\item{} {\bf Wright, P. J.}, Hannah, I.~G., Viall, N.~M., \textit{et al} 2019. {\it The Thermal Time Evolution of Active Regions Determined by SDO/AIA}
	\item{} {\bf Wright, P. J.}, Saar, S.~H., Meibom, S., \textit{et al} 2019. {\it  The Age-Dependent Superflare Rates of G-Type Dwarfs In Three {\it Kepler} Clusters}
	%\item{} {\bf Wright, P. J.}, Saar, S.~H., Meibom, S., \textit{et al} 2019. {\it An Extension of The Age-Dependent Superflare Rates to F- and K-Type Dwarfs}
\end{bibenum}

\section{Conferences, Workshops, \& Schools}

\textbf{Invited Oral Presentations}
\begin{innerlist}
     \item[] \textit{ISSI Team Meeting: Coronal Nanoflares}, Bern, CH \hfill {2018} % \hfill {Jan. 2016}
     \item[] \textit{ISSI Team Meeting: Coronal Nanoflares}, Bern, CH \hfill {2016} % \hfill {Jan. 2016}
     \item[] \textit{Harvard-Smithsonian Center for Astrophysics}, Cambridge, MA, USA \hfill {2014} % \hfill {May 2014}
\end{innerlist}

\halfblankline

\textbf{Oral/e-Poster Presentations}
\begin{innerlist}
     \item[] \textit{Living with a Star (SDO/LWS) Workshop}, Ghent, Belgium \hfill {2018} % \hfill {Sept. 2015}
     \item[] \textit{Solar Physics Division Meeting (SPD/AAS)}, Portland, OR, USA \hfill {2017} % \hfill {Sept. 2015}
     \item[] \textit{Coronal Loops Workshop \textsc{viii}}, Palermo, Sicily, IT \hfill {2017} % \hfill {Sept. 2015}
     \item[] \textit{Living with a Star (SDO/LWS) Workshop}, Burlington, VT, USA \hfill {2016} % \hfill {Sept. 2015}
     \item[] \textit{Hinode 10}, Nagoya, JP \hfill {2016} % \hfill {Sept. 2015}
     \item[] \textit{National Astronomy Meeting 2016}, Nottingham, UK \hfill {2016} % \hfill {Sept. 2015}
     \item[] \textit{Hinode 9}, Belfast, UK \hfill {2015} % \hfill {Sept. 2015}
%\end{innerlist} 
%
%\newpage
%\makeheading{Paul James Wright} 
%
%
%\section{Conferences, Workshops, \& Schools (Cont.)}
%\textbf{Oral/e-Poster Presentations (Cont.)}
%\begin{innerlist}
     \item[] \textit{Glasgow-Cambridge Flare Workshop}, Glasgow, UK \hfill {2015} % \hfill {Apr. 2015}
\end{innerlist}

\halfblankline

\textbf{Poster Presentations}
\begin{innerlist}
     \item[] \textit{European Solar Physics Meeting (ESPM)}, Budapest, HU \hfill {2017}
     \item[] \textit{Solar Physics Division Meeting (SPD/AAS)}, Portland, OR, USA \hfill {2017} % \hfill {Sept. 2015}
     \item[] \textit{Living with a Star (SDO/LWS) Workshop}, Burlington, VT, USA \hfill {2016} % \hfill {Sept. 2015}
     \item[] \textit{Coronal Loops Workshop \textsc{vii}}, Cambridge, UK \hfill {2015} % \hfill {July 2015}
     \item[] \textit{National Astronomy Meeting (NAM) 2015}, Llandudno, UK \hfill {2015} %\hfill {July 2015}
     \item[] \textit{223rd AAS Meeting}, National Harbor, MD, USA \hfill {2014} %\hfill {Jan. 2014}
\end{innerlist}

\halfblankline

\textbf{Schools Attended}
\begin{innerlist}
     \item[] \textit{CESRA Radio Summer School 2015}, Glasgow, UK \hfill {2015} %\hfill {Aug. 2015}
     \item[] \textit{STFC Advanced Summer School in Solar Physics}, Dundee, UK \hfill {2014} %\hfill {Aug. -- Sept. 2014}
\end{innerlist}

\halfblankline

\textbf{Additional Conferences/Workshops Attended}
\begin{innerlist}
     \item[] \textit{NuSTAR Heliophysics Workshop (remote participation)}, Berkeley, CA, USA \hfill {2017} %\hfill {Nov. 2014}
     \item[] \textit{SUPA Cormack Astronomy Meeting}, Edinburgh, UK \hfill {2015} %\hfill {Nov. 2014}
     \item[] \textit{Royal Astronomical Society Discussion Meeting: Results from IRIS}, London, UK \hfill {2015} %\hfill {Jan. 2015}
     \item[] \textit{SUPA Cormack Astronomy Meeting}, Edinburgh, UK \hfill {2014} %\hfill {Nov. 2014}
     \item[] \textit{1st Space Glasgow Research Conference}, Glasgow, UK \hfill {2014} %\hfill {Oct. 2014}
\end{innerlist}

\section{Awards and Grants \\ Total: \textsterling7000}

{\bf University of Glasgow}
\begin{innerlist}
\item[] \href{}{NASA Frontier Development Lab, Contribution to Science Award} \hfill 2018
\item[] \href{https://spd.aas.org/education/students#poster_award}{Solar Physics Division Meeting (AAS/SPD) Student Poster Award} \hfill 2017
\item[] \href{https://spd.aas.org/prizes/studentship}{Solar Physics Division Meeting (AAS/SPD) Studentship Award} \hfill 2017
\item[] \href{}{Coronal Loops Workshop \textsc{viii} Travel Award } \hfill 2017
\item[] \href{}{National Astronomical Observatory of Japan Travel Award} \hfill 2016
\item[] \href{}{Hinode 9 Travel Award} \hfill 2015
\item[] \href{}{European Space Agency/Cambridge Philosophical Society Travel Award} \hfill 2015
\end{innerlist}
\end{innerlist}
\end{innerlist}

\clearpage

\end{innerlist}
%
\newpage
\makeheading{Paul James Wright} 
%{\bf Harvard University/Harvard-Smithsonian CfA}
%\begin{innerlist}
%\item[] \href{}{223rd AAS Travel Grant} \hfill 2014
%\end{innerlist}
%


\section{Awards and Grants (Cont.)}
{\bf University of Southampton}
\begin{innerlist}
\item[] \href{}{Research Scholarship} \hfill 2013
\item[] \href{}{Summer Studentship Grant} \hfill 2013
\end{innerlist}
\end{innerlist}

\section{Teaching}
\href{http://www.coursera.org/}{\textbf{Coursera Inc.}}
\begin{innerlist}
\item[] \href{http://www.coursera.community/#mentor}{\textbf{``Data Scientists Toolbox'' Community Mentor}} \hfill {2017 -- present}
\begin{innerlist}
\item[] An invited mentor of a course in the Data Science specialisation offered by Johns Hopkins University. %Mentors are invited based on academic performance and by the support provided to fellow learners in the course forums.
\end{innerlist}
\end{innerlist}
\halfblankline

\href{http://www.gla.ac.uk/}{\textbf{University of Glasgow}}
\begin{innerlist}
\item[] \textbf{Astronomy 1 Tutorial Demonstrator} \hfill {2016 - 2017}
\begin{innerlist}
\item[] Supervised students, and marked first-year astronomy problem sets.
\end{innerlist}
\item[] \textbf{Astronomy 3/4 (Honours) Laboratory Demonstrator} \hfill {2015 - 2016}
\begin{innerlist}
\item[] Demonstrated, supervised, and marked a number of final-year research projects covering topics such as asteroid light curves, and solar limb darkening.
\end{innerlist}
\item[] \textbf{Physics Pre-University Summer School} \hfill {2015} 
\begin{innerlist}
\item[] Taught at a pre-university school for students entering the first year of undergraduate education.
\end{innerlist}
\end{innerlist}

\section{Memberships}
     \href{https://www.nustar.caltech.edu/page/sun}{\textbf{\textit{NuSTAR} Heliophysics Working Group}}, Member \hfill {2015 -- present} \\
     \href{http://www.issibern.ch/}{\textbf{International Space Science Institute (ISSI)}}, Young Scientist Member \hfill {2015 -- present}
    \begin{innerlist}
     \item[] Member of Paola Testa's ISSI Team: \href{http://helio.cfa.harvard.edu/ISSI/welcome_pt.html}{\textit{New Diagnostics of Particle Acceleration in Solar Coronal Nanoflares from Chromospheric Observations and Modelling}}
    \end{innerlist}
    \href{https://www.ras.org.uk/}{\textbf{Royal Astronomical Society}}, RAS Fellow \hfill {2014 -- present}
    % \item American Geophysical Union (AGU Member) \hfill {2015 -- present}
% The ``More Info'' section may not be necessary; make sure it's short
% so it doesn't prevent people from seeing references available to
% contact.

  \end{innerlist}
%\section{Technical Skills:} Python, IDL, $\LaTeX$, Linux/Unix, Bash, Microsoft Office
%{\it General:} Interdisciplinary Collaboration, Public Speaking, Teaching, Writing (Technical \& Lay), Data Analysis, Statistics

\section{Community Involvement}

%\begin{innerlist}
{\bf \href{http://www.nature.com/ncomms/}{Nature Communications}}, Reviewer \hfill 2017 -- present \\
{\bf \href{}{Glasgow Astronomy \& Astrophysics Group Meeting}}, Organiser \hfill 2017 \\
{\bf \href{http://www.astro.gla.ac.uk/?page_id=3673}{CESRA Radio Summer School}}, Volunteer Organiser \hfill 2015
%\end{innerlist}

\section{Scientific Outreach}

%\begin{innerlist}
{\bf \href{http://www.glasgowsciencecentre.org/}{Glasgow Science Centre}}, Demonstrator \hfill 2016  \\
{\bf \href{https://www.britishscienceweek.org/}{British Science Week}}, Demonstrator \hfill 2016 \\
{\bf \href{http://www.gla.ac.uk/explore/scienceconnects/news/headline_451512_en.html}{Institute of Physics: {\it Women and Girls in Science}}}, Demonstrator \hfill 2016 \\
{\bf \href{http://www.stv.tv/}{Scottish Television (STV)}}, Guest Presenter \hfill 2015 \\
{\bf \href{http://www.worldwidetelescope.org/}{World Wide Telescope}}, Ambassador \hfill 2013 -- 2014 \\
{\bf \href{http://www.bbc.co.uk/programmes/b019h4g8}{BBC Stargazing Live}}, Demonstrator \hfill 2013 \\
{\bf \href{http://www.southampton.ac.uk/~sj2y10/}{So'ton Astrodome}}, Demonstrator \hfill 2012 \\
{\bf \href{http://www.bbc.co.uk/programmes/b00lwxj1}{BBC Bang Goes The Theory Roadshow}}, Demonstrator \hfill 2012

\halfblankline

{\bf {UK Solar Physics (UKSP) Nuggets}}, concise, easy-to-read science articles
\begin{innerlist}
\item[] {\href{http://www.uksolphys.org/uksp-nugget/84-the-first-nustar-microflare}{84. The first {\it NuSTAR} microflare} \hfill 2017
\end{innerlist}

\halfblankline

{\bf {Hinode/XRT Picture of the Week (XPOW)}}
\begin{innerlist}
\item[] {\href{http://xrt.cfa.harvard.edu/xpow/20171031.html}{The First Microflare Observations with {\it Hinode}/XRT \& {\it NuSTAR}} \hfill 2017
\end{innerlist}
\end{innerlist}

\section{Personal Projects} {\bf \href{https://github.com/PaulJWright/ColourBlind}{ColourBlind}}, A repository for colour-blind-friendly colour tables. %\hfill Citations: 1

\newpage
\makeheading{Paul James Wright} 

\section{Professional Development}
{\bf{Coursera, Inc. (MOOC Platform)}} \\
{Using Coursera.org, a massive open online course (MOOC) platform, to take specialisations (a series of related courses plus a final capstone project) offered by accredited universities to further develop skills and understanding in a wide range of topics.} \\
%{\bf{Massive Open Online Course}}:
\begin{innerlist}
    \item[] {\bf{\href{https://www.coursera.org/specializations/jhu-data-science}{Data Science}}}, Johns Hopkins University \hfill{2017 -- present}
    \begin{innerlist}
    	\item[] Nine-course (plus capstone) introduction to data science.
    \end{innerlist}
    \item[] {\bf{\href{https://www.coursera.org/specializations/r}{Mastering Software Development in R}}}, Johns Hopkins University \hfill{2018 -- present}
    \begin{innerlist}
    	\item[] Four-course (plus capstone) specialisation providing rigorous training in R.
    \end{innerlist}
    %\item[] {\bf{\href{https://www.coursera.org/specializations/statistics}{Statistics with R}}}, Duke University \hfill{2018 -- present}
    %\begin{innerlist}
    %	\item[] Four-course (plus capstone) specialisation providing further training in R, with emphasis on statistics.
    %\end{innerlist}    
\end{innerlist}    



%\section{Professional Development (Cont.)}    
%\begin{innerlist}
%    \item[] {\bf{\href{https://www.coursera.org/specializations/big-data}{Big Data}}}, UC San Diego \hfill{2017 -- present}
%    \begin{innerlist}
%    	\item[]  Five-course (plus capstone) introduction to big data using Hadoop with MapReduce, Spark, Pig and Hive.
%    \end{innerlist}        
%    \item[] {\bf{\href{https://www.coursera.org/specializations/machine-learning}{Machine Learning}}}, University of Washington \hfill{2017 -- present}
%    \begin{innerlist}
%    	\item[] Three-course (plus capstone) introduction to Machine Learning.
%    \end{innerlist}       
%    \item[] {\bf{\href{https://www.coursera.org/specializations/graphic-design}{Graphic Design}}}, CalArts \hfill{2017 -- present}
%    \begin{innerlist}
%    	\item[] Four-course (plus capstone) introduction the fundamental skills required to make sophisticated graphic design. \\
 %   \end{innerlist}
 \end{innerlist}

%\newpage
%\makeheading{Paul James Wright} 

%
%\section{Professional Development (Cont.)}    
%{\bf{edx, Inc. (MOOC Platform)}}
%%{\bf{Massive Open Online Course}}:
%    \begin{innerlist}
%    \item[] {\bf{\href{https://www.edx.org/course/introduction-computer-science-harvardx-cs50x#!}{Introduction to Computer Science (CS50x)}}}, Harvard University \hfill{2017 -- present}
%    \begin{innerlist}
%    \item[] An introduction to the intellectual enterprises of computer science and the art of programming including languages such as C, and SQL.
%    \end{innerlist}
               
% Courses I want to address
      
%   \item[] {\it{Specializations/Series}}: \href{https://www.coursera.org/specializations/big-data}{Big Data (UC San Diego)}; \href{https://www.coursera.org/specializations/jhu-data-science}{Data Science (Johns Hopkins)}; \href{https://www.coursera.org/specializations/statistics}{Statistics with R (Duke)}; \href{https://www.edx.org/microsoft-data-science-curriculum}{Data Science (Microsoft)}; \href{https://digitalgarage.withgoogle.com}{Google Digital Garage}; \href{https://academy.microsoft.com/en-us/professional-program/data-science/}{Microsoft Professional Program Certificate in Data Science}; \href{https://www.edx.org/course/machine-learning-columbiax-csmm-102x}{Machine Learning (Columbia)}
%   \item[] {\it{Single Courses}}: \href{https://www.coursera.org/learn/machine-learning}{Machine Learning (Stanford)}; \href{https://www.coursera.org/learn/bayesian-statistics}{Bayesian Statistics (UC Santa Cruz)}; \href{https://www.edx.org/course/introduction-computer-science-harvardx-cs50x}{Introduction to Computer Science (CS50x; Harvard)}; \href{https://www.edx.org/xseries/genomics-data-analysis}{Genomics Data Analysis; Harvard}; \href{https://www.edx.org/course/machine-learning-columbiax-csmm-102x}{Machine Learning (Columbia)}; \href{https://www.edx.org/course/statistical-inference-modeling-high-harvardx-ph525-3x}{Statistical Inference and Modeling for High-throughput Experiments}; \href{https://www.edx.org/course/high-dimensional-data-analysis-harvardx-ph525-4x}{High-Dimensional Data Analysis} \href{https://www.edx.org/course/introduction-computational-thinking-data-mitx-6-00-2x-5}{Introduction to Computational Thinking and Data Science}
    \end{innerlist}

% \section{Technical Skills:} {\it Computing:} C, Python, R (caret, ggplot2, knitr), SQL, CRAN, IDL, $\LaTeX$, git, GitHub, Hadoop (MapReduce, Spark, Pig, Hive), Linux/Unix, Mac OSX, Microsoft Windows, Bash, Microsoft Office, Adobe Creative Cloud, Keynote, Wordpress, Shiny, GoogleVis, and Plotly, HTML, CSS, Javascript
% \\ \\
%{\it General:} Data Analysis, Data Visualization, Interdisciplinary Collaboration, Public Speaking, Statistics, Teaching, Writing (Technical \& Lay)

\section{Technical Skills:} {\it Computing:} IDL (5+ years), Python (2+ years), PyTorch, R, Bash, $\LaTeX$, PyCharm, IRAF, git (GitHub, Gitlab), Microsoft Office, Adobe Creative Cloud, Linux/Unix, Mac OSX, Microsoft Windows

\halfblankline

{\it General:} Data Analysis, Data Visualisation, Interdisciplinary Collaboration, Public Speaking, Teaching, Writing (Technical \& Lay)

%\section{References}
%Dr Iain Hannah \\
%\begin{innerlist}
%\item{} Rm 620, SUPA School of Physics \& Astronomy, \\ Kelvin Building, \\ University of Glasgow, \\ Glasgow, \\ G12 8QQ, \\ UK
%\item{} Email: iain.hannah@glasgow.ac.uk
%\item{} Office Phone: +44 141 330 6427
%\end{innerlist}    
%Dr Alexander MacKinnon \\
%\begin{innerlist}
%\item{} Rm 620, SUPA School of Physics \& Astronomy, \\ Kelvin Building, \\ University of Glasgow, \\ Glasgow, \\ G12 8QQ, \\ UK
%\item{} Email: iain.hannah@glasgow.ac.uk
%\item{} Office Phone: +44 141 330 6427
%\end{innerlist}    
%Dr Steven Saar \\
%\begin{innerlist}
%\item{} Rm 620, SUPA School of Physics \& Astronomy, \\ Kelvin Building, \\ University of Glasgow, \\ Glasgow, \\ G12 8QQ, \\ UK
%\item{} Email: iain.hannah@glasgow.ac.uk
%\item{} Office Phone: +44 141 330 6427
%\end{innerlist}    

        
%\section{More Information}

%More information and auxiliary documents can be found at \href{http://www.pauljwright.co.uk}{http://www.pauljwright.co.uk}, on \href{https://www.researchgate.net/profile/Paul_Wright7}{ResearchGate}, and \href{http://www.github.com/pauljwright}{GitHub}.

%\section{References}    
%\begin{innerlist}
%    \item[] {\bf{Dr Iain G. Hannah}} \\ SUPA School of Physics and Astronomy \\ University of Glasgow \\ Glasgow, G12 8QQ, UK \\ iain.hannah@glasgow.ac.uk \\
%    \item[] {\bf{Dr Steven Saar}} \\ Harvard-Smithsonian Center for Astrophysics \\ 60 Garden Street \\ Cambridge, MA 02138, USA  \\ ssaar@cfa.harvard.edu \\
%\end{innerlist}
\end{document}

%%%%%%%%%%%%%%%%%%%%%%%%%% End CV Document %%%%%%%%%%%%%%%%%%%%%%%%%%%%%

%----------------------------------------------------------------------%
% The following is copyright and licensing information for
% redistribution of this LaTeX source code; it also includes a liability
% statement. If this source code is not being redistributed to others,
% it may be omitted. It has no effect on the function of the above code.
%----------------------------------------------------------------------%
% Copyright (c) 2007, 2008, 2009, 2010, 2011 by Theodore P. Pavlic
%
% Unless otherwise expressly stated, this work is licensed under the
% Creative Commons Attribution-Noncommercial 3.0 United States License. To
% view a copy of this license, visit
% http://creativecommons.org/licenses/by-nc/3.0/us/ or send a letter to
% Creative Commons, 171 Second Street, Suite 300, San Francisco,
% California, 94105, USA.
%
% THE SOFTWARE IS PROVIDED "AS IS", WITHOUT WARRANTY OF ANY KIND, EXPRESS
% OR IMPLIED, INCLUDING BUT NOT LIMITED TO THE WARRANTIES OF
% MERCHANTABILITY, FITNESS FOR A PARTICULAR PURPOSE AND NONINFRINGEMENT.
% IN NO EVENT SHALL THE AUTHORS OR COPYRIGHT HOLDERS BE LIABLE FOR ANY
% CLAIM, DAMAGES OR OTHER LIABILITY, WHETHER IN AN ACTION OF CONTRACT,
% TORT OR OTHERWISE, ARISING FROM, OUT OF OR IN CONNECTION WITH THE
% SOFTWARE OR THE USE OR OTHER DEALINGS IN THE SOFTWARE.
%----------------------------------------------------------------------%
